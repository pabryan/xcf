\documentclass{amsart}
\usepackage[ocgcolorlinks,linktoc=all]{hyperref}
\usepackage{cancel}
\hypersetup{citecolor=blue,linkcolor=red}
\newtheorem{theorem}{Theorem}
\newtheorem*{thmA}{Theorem}
\newtheorem*{thmB}{Theorem}
\newtheorem*{rem}{Remark}
\newtheorem*{thmmain}{Theorem}
\newtheorem{lemma}[theorem]{Lemma}
\newtheorem{proposition}[theorem]{Proposition}
\newtheorem*{propmain}{Proposition}
\newtheorem{corollary}[theorem]{Corollary}
\theoremstyle{definition}
\newtheorem{definition}[theorem]{Definition}
\newtheorem{example}[theorem]{Example}
\newtheorem{xca}[theorem]{Exercise}

\theoremstyle{remark}
\newtheorem{remark}[theorem]{Remark}

\newcommand{\abs}[1]{\lvert#1\rvert}
\numberwithin{equation}{section}

\newcommand{\blankbox}[2]{%
  \parbox{\columnwidth}{\centering
    \setlength{\fboxsep}{0pt}%
    \fbox{\raisebox{0pt}[#2]{\hspace{#1}}}%
  }%
}

\begin{document}

\title[Cross curvature flow]
 {Cross curvature flow}

\curraddr{}
\email{}
\date{\today}

\dedicatory{}
\subjclass[2010]{}
\keywords{}

\begin{abstract}
\end{abstract}

\maketitle

\section{Notation}
The Riemannian metric is $g_{ij},$ its inverse is $g^{ij}$. The Levi-Civita connection is given by the Christoffel symbols
\begin{equation}
\Gamma_{ij}^k=\frac{1}{2}g^{kl}\left(\partial_ig_{jl}+\partial_jg_{il}-\partial_lg_{ij}\right).
\end{equation}
Write $\nabla^2_{i,j}$ for the second covariant derivative; for any tensor $\alpha:$
\[\nabla^2_{i,j}\alpha=\nabla_{\partial_i}\nabla_{\partial_j}\alpha-\nabla_{\nabla_{\partial_i}\partial_j}\alpha.\]
The Riemannian curvature tensor is
\begin{equation}
R_{ijk}^l=\partial_i\Gamma_{jk}^l-\partial_j\Gamma_{ik}^l+\Gamma_{jk}^p\Gamma_{ip}^l-\Gamma_{ik}^p\Gamma_{jp}^l.
\end{equation}
We lower the index to the third position, so that
\begin{equation}
R_{ijkl}=g_{kp}R_{ijl}^p.
\end{equation}
The Ricci curvature is the contraction
\begin{equation}
R_{ij}=g^{kl}R_{ikjl}.
\end{equation}
The scalar curvature $R=g^{ij}R_{ij}.$


The evolution equation of the metric is given by
\begin{equation}
\partial_tg_{ij}=2h_{ij},
\end{equation}
where
\begin{equation}
h_{ij}:=-\frac{1}{2}P^{kl}R_{ikjl}
\end{equation}
and
\begin{equation}
P_{ij}:=R_{ij}-\frac{1}{2}Rg_{ij},\quad
P^{mn}=\left(R_{ij}-\frac{1}{2}Rg_{ij}\right)g^{im}g^{jn}.
\end{equation}
Let $V_{ij}$ be the inverse of $P^{ij}$. We mention that
\begin{equation}
h_{ij}=V_{ij}\det P,
\end{equation}
where $\det P = \det (g_{km} P^{ml}).$


We also write
\begin{align}
P&=g_{ij}P^{ij},\quad V=g^{ij}V_{ij},\quad H=g^{ij}h_{ij},\quad u:=\log\det P\\
T^{kij}&=P^{kl}\nabla_l P^{ij},\quad T^i=V_{jk}T^{ijk}=P^{ij}\nabla_ju.
\end{align}
In the sequel, $\mu_{ijk}$ is the volume form and is nonzero only when $i,j,k$ are distinct, in which case it is sign of the permutation $(ijk)$. Using the volume form, we may express $P^{mn}$ as
\begin{equation}
P^{mn}=-\frac{1}{4}\mu^{ijm}\mu^{kln}R_{ijkl}.
\end{equation}
We conclude this section by scatting the following identities from \cite[Lemma 1]{Chowcross2002}:
\begin{align}\label{equ: important}
P^{ij}\nabla_ih_{jk}&=\frac{1}{2}P^{ij}\nabla_kh_{ij},\quad
\nabla_iP^{ij}=0.
\end{align}
\section{Solitons}
A solution to cross curvature flow is a cross curvature soliton if there exists a vector field $W$ and $\lambda\in \mathbb{R}$ such that at some time
\begin{equation}\label{soliton 0}
2\lambda g_{ij}=2h_{ij}+\nabla_iW_j+\nabla_jW_i.
\end{equation}
For an expanding soliton, (\ref{soliton 0}) holds with $\lambda=\frac{1}{4t}.$
To obtain a suitable Harnack quadratic for XCF, we follow Hamilton's procedure in obtaining his trace Harnack quantity for the Ricci flow.


We take the covariant derivative of both sides of (\ref{soliton 0}) with respect to $\nabla_l:$
\begin{align}\label{eq: soliton derv}
2\nabla_lh_{ij}+\nabla^2_{l,i}W_j+\nabla^2_{l,j}W_i=0.
\end{align}
We also take the trace of (\ref{soliton 0}) with respect to $P^{ij}$:
\begin{align}\label{trace}
\lambda P=3\det P+P^{ij}\nabla_iW_j.
\end{align}
Taking the covariant derivative of both sides with respect to $\nabla_l$ yields
\begin{align*}
\lambda \nabla_l P=3\nabla_l\det P+\nabla_l P^{ij}\nabla_iW_j+P^{ij}\nabla^2_{l,i}W_j.
\end{align*}
In view of
\begin{align*}
\nabla^2_{l,i}W_j-\nabla^2_{i,l}W_j=R_{lijk} g^{km}W_m,
\end{align*}
we obtain
\begin{align}\label{equ00}
\lambda \nabla_l P&=3\nabla_l\det P+\nabla_l P^{ij}\nabla_iW_j+P^{ij}(\nabla^2_{i,l}W_m -R_{likj}g^{km}W_m)\\
&=3\nabla_l\det P+\nabla_l P^{ij}\nabla_iW_j+P^{ij}\nabla^2_{i,l}W_j+2h_l^mW_m.\nonumber
\end{align}
For now on  we consider gradient solitons which are by definition, when $W_i=\nabla_if$ for a smooth function (note that in this case $\nabla_iW_j=\nabla_jW_i$).
Putting (\ref{soliton 0}), (\ref{eq: soliton derv}) together we obtain
\begin{align}\label{eq:1}
\lambda \nabla_l P
=&3\nabla_l\det P+(\lambda g_{ij}-h_{ij})\nabla_l P^{ij}-P^{ij}\nabla_ih_{jl}+2h_l^mW_m\\
=&3\nabla_l\det P+\lambda \nabla_l P-h_{ij}\nabla_l P^{ij}-\frac{1}{2}P^{ij}\nabla_lh_{ij}+2h_l^mW_m.\nonumber
\end{align}
We may rewrite (\ref{eq:1}) as
\[3\nabla_l\det P+2h_l^mW_m-h_{ij}\nabla_l P^{ij}-\frac{1}{2}P^{ij}\nabla_lh_{ij}=0.\]
We divide both sides by $\det P:$
\begin{equation}\label{eq:a}
\nabla_lu+\frac{2}{\det P}h_l^mW_m=0,
\end{equation}
where we used
\begin{equation}\label{eq0}
\frac{P^{ij}}{\det P}\nabla_lh_{ij}=2\nabla_lu,\quad V_{ij}\nabla_lP^{ij}=\nabla_lu.
\end{equation}
Taking the covariant derivative of (\ref{eq:a}) with respect to $\nabla_k$ and then using the first soliton equation (\ref{soliton 0}) yields
\begin{equation*}
\nabla^2_{k,l}u+\frac{2}{\det P}\nabla_kh_l^mW_m-\frac{2}{\det P}h_l^mW_m\nabla_ku-2V_l^mh_{km}+2\lambda V_{kl}=0.
\end{equation*}
We take the trace with respect to $P^{kl}$ and use the first identity in (\ref{eq0}):
\begin{equation}\label{eq:5}
\Box u-2H+6\lambda=0.
\end{equation}
%We take the trace with respect to $P^{kl}:$
%\begin{equation}\label{eq:4}
%\Box u+P^{kl}\nabla_ku\nabla_lu+2\nabla_kuV^k-2H+6\lambda=0.
%\end{equation}
%On the other hand, there holds
%\begin{align}\label{first derv test}
%\nabla^mh_{ij}-\nabla_jh_i^m={{{R_j}^{m}}_{i}}^nV_n\Rightarrow 4V_{kl}V^kV^l+2\nabla_kuV^k=0.
%\end{align}
%Hence adding (\ref{first derv test}) to (\ref{eq:4}) implies that
%\begin{align}\label{harnack quant}
%\Box u+4\left|V^i+\frac{1}{2}P^{ij}\nabla_ju\right|_{V_{kl}}^2
%-2H+6\lambda=0.
%\end{align}
So Lemma \ref{ev u} proposes the following candidate as the  Harnack inequality:
 \begin{align}\label{harnack est}
\partial_tu-\frac{1}{2}\left(\left|E^{ijk}-E^{jik}\right|^2_V+\left|T^i\right|_V^2\right)+\frac{3}{2t}\geq 0,
\end{align}
 where
 \[T^{ijk}=E^{ijk}-\frac{1}{10}\left(P^{ij}T^k+P^{ik}T^j\right)+\frac{2}{5}P^{jk}T^i,\]
 and we have $T^{ijk}V_{ij}=0,$ $V_{ij}E^{ijk}=V_{ik}E^{ijk}=V_{jk}E^{ijk}=0;$ see \cite[Prop. 9]{Chowcross2002}. In the case of embeddability into Minkowski 4-space, we have
 \begin{align*}
\left|E^{ijk}-E^{jik}\right|^2_V=0,\quad\frac{1}{2}\left|T^i\right|_V^2=\frac{1}{2}P^{ij}\nabla_i u\nabla_ju.
 \end{align*}
In fact, $E^{ijk}=E^{jik}$ due to the Codazzi equation which yields the full symmetry of the covariant derivative of the second fundamental form. In general, if 
$\alpha_{ij}:=\sqrt{\det P}V_{ij}$ satisfies
$\nabla_k\alpha_{ij}=\nabla_i\alpha_{jk},$
then \[E^{ijk}=E^{jik}.\]
\begin{theorem}
Let $(M^3,g(t))$, $t\in [0,T)$, be a solution to the XCF on a closed 3-manifold starting from a metric $g(0)$ with negative sectional curvature. Then $g(t)$ has negative sectional curvature for all $t<T.$
Moreover, it is not possible that $T<\infty$ and $\inf_{M\times [0,T)}\det P=0.$
\end{theorem}
\begin{proof}
Suppose $0<t_{\ast}<T$ is the first time that $g(t_{\ast})$ has a zero sectional curvature at some point $x_{\ast}$. At $(x_{\ast},t_{\ast})$ we must have $\det P(x_{\ast},t_{\ast})=0.$
By our Harnack inequality, we know that $t^{\frac{3}{2}}\det P(\cdot,t) $ is increasing in time on $(0,t_{\ast})$; therefore, for any $t\in [t_{\ast}/2, t_{\ast})$ we have
\[\det P(\cdot,t)\geq \det P(\cdot,t_{\ast}/2)\left(\frac{1}{2}\right)^{\frac{3}{2}}>0.\]
Thus by continuity of the solution we obtain $\det P(\cdot,t_{\ast})>0$, which yields a contradiction.

By the first part of the theorem, $P^{ij}$ is invertible on $[t_1,t_2]$ for any $t_1,t_2$ with $0<t_1\leq t_2<T.$ Thus the Harnack inequality is available on $[t_1,t_2]$ and it gives
\[\det P(\cdot,t_2)\geq \det P(\cdot,t_1)\left(\frac{t_1}{t_2}\right)^{\frac{3}{2}}.\]
Allowing $t_2\to T$ completes the proof.
\end{proof}
Before moving on to the proof of Harnack inequality, we give a classification of compact solitons of XCF with negative sectional curvatures.
\begin{theorem}
The only compact solitons of XCF with negative sectional curvature are metrics with constant negative curvature.
\end{theorem}
\begin{proof}
Apply (\ref{soliton 0}) to (\ref{equ00}):
\begin{align*}
\lambda \nabla_l P&=3\nabla_l\det P+(\lambda g_{ij}-h_{ij})\nabla_l P^{ij}+P^{ij}\nabla^2_{i,l}W_j+2h_l^mW_m\\
&=3\nabla_l\det P+\lambda \nabla_l P-h_{ij}\nabla_l P^{ij}+P^{ij}(-2\nabla_ih_{lj}-\nabla^2_{i,j}W_l)+2h_l^mW_m.
\end{align*}
Dividing both sides by $\det P$ implies that
\begin{align*}
3\nabla_lu-V_{ij}\nabla_l P^{ij}-\frac{P^{ij}}{\det P}\nabla_lh_{ij}-\frac{1}{\det P}(\Box W_l-2h_l^kW_k)=0.
\end{align*}
Therefore, by (\ref{eq0}) we obtain
\begin{align*}
\Box W_l-2h_l^kW_k=0\Rightarrow W^l\Box W_l-2h_l^kW_kW^l=0.
\end{align*}
The second identity reads
\begin{align}\label{soliton equation for W}
\Box \frac{1}{2}|W|_g^2-P^{kl}g^{ij}\nabla_kW_i\nabla_lW_j-2h^{kl}W_kW_l=0.
\end{align}
Note that, by (\ref{equ: important}), $\int \Box fd\mu_{M}=0$ for any smooth function $f$ defined on $M$. Thus integrating (\ref{soliton equation for W}) against $d\mu_{M}$ and taking into account that $P_{ij}$ and $h_{ij}$ are both positive definite proves that $W\equiv0$ (also note that $g,P$ and $h$ can all be diagonalized simultaneously at any fixed point); therefore, by (\ref{soliton 0}), we get
\[(\det P)^2=\det h_{ij}=\lambda ^3\Rightarrow \det P=\lambda^{\frac{3}{2}}.\]
Moreover, in view of (\ref{trace}), we have $3\det P=\lambda P$; therefore, we obtain $$3(\det P)^{\frac{1}{3}}=P.$$ Thus the metric has constant negative curvature.
\end{proof}
\section{evolution equations}
We first collect a few identities that will be used without further mention:
\begin{align}\label{equ: important1}
P^{ij}\nabla_ih_{jk}&=\frac{1}{2}P^{ij}\nabla_kh_{ij},\quad
\nabla_iP^{ij}=0,\quad \nabla_iT^i=\Box u,\\
T^k\nabla_kP^{ij}&=T^{kij}\nabla_ku,\quad P^{ij}\nabla_j T^k=P^{ij}P^{kl}\nabla^2_{j,l}u+T^{ikl}\nabla_lu,\\
T^{ijk}-T^{jik}&=E^{ijk}-E^{jik}+\frac{1}{2}\left(P^{jk}T^i-P^{ik}T^j\right).
\end{align}
For simplicity, write
\[D^{ijk}=E^{ijk}-E^{jik}.\]
The tensor $D^{ijk}$ satisfies
\begin{align*}
D^{ijk}=-D^{jik},\quad D^{ijk}+D^{kij}+D^{jki}=0.
\end{align*}
\begin{lemma}\label{ev u}
The following evolution equations hold:
\begin{align*}
\partial_tP^{mn}
=&-g^{mn}\det P-HP^{mn}+\frac{1}{4}P^{mn}T^k\nabla_ku\\
&-\frac{1}{4}T^mT^n+\frac{1}{2}P^{mn}\Box u-\frac{1}{2}P^{mk}P^{nl}\nabla^2_{k,l}u\\
&+\nabla_kD^{knm}+\frac{1}{2}D^{kmn}\nabla_ku,
\end{align*}
\begin{align*}
\partial_t u=&\Box u+\frac{1}{2}\left(\left|D^{ijk}\right|^2_V+\left|T^i\right|_V^2\right)-2H.
\end{align*}
\end{lemma}
\begin{proof}
The evolution equation of the Riemann curvature tensor is given by the standard formula
\begin{align*}
\partial_t R_{ijkl}=&\nabla^2_{i,l}h_{jk}+\nabla^2_{j,k}h_{il}-\nabla^2_{i,k}h_{jl}-\nabla^2_{j,l}h_{ik}\\
&+g^{pq}(R_{ijpk}h_{ql}+R_{ijpl}h_{qk}).
\end{align*}
On the other hand, the evolution of the volume form is given by $$\partial_t\mu_{ijk}=H\mu_{ijk},\quad\partial_t\mu^{ijk}=-H\mu^{ijk}.$$
Since $P^{mn}=-\frac{1}{4}\mu^{ijm}\mu^{kln}R_{ijkl}$, we obtain
\begin{align*}
\partial_tP^{mn}=&\mu^{ijm}\mu^{kln}\nabla^2_{i,k}h_{jl}-\frac{1}{2}\mu^{ijm}\mu^{kln}g^{pq}R_{ijpl}h_{qk}-2HP^{mn}.
\end{align*}
Now using the identity
\[\frac{1}{2}\mu^{ijm}\mu^{kln}g^{pq}R_{ijpl}h_{qk}+HP^{mn}=g^{mn}\det P,\]
we arrive at
\begin{align*}
\partial_tP^{mn}=&\nabla^2_{k,l}(P^{kl}P^{mn}-P^{km}P^{ln})-g^{mn}\det P-HP^{mn}\\
=&\nabla_k(P^{kl}\nabla_lP^{nm}-P^{nl}\nabla_lP^{km})-g^{mn}\det P-HP^{mn}\\
=&\nabla_k\left(T^{knm}-T^{nkm}\right)-g^{mn}\det P-HP^{mn}\\
=&\nabla_kD^{knm}+\frac{1}{2}\nabla_k\left(P^{nm}T^k-P^{mk}T^n\right)-g^{mn}\det P-HP^{mn}\\
=&\nabla_kD^{knm}-g^{mn}\det P-HP^{mn}+\frac{1}{2}P^{mn}\Box u\\
&+\frac{1}{2}\left(T^{kmn}-T^{mkn}\right)\nabla_ku-\frac{1}{2}P^{mk}P^{nl}\nabla^2_{k,l}u\\
=&\nabla_kD^{knm}+\frac{1}{2}D^{kmn}\nabla_ku-g^{mn}\det P-HP^{mn}+\frac{1}{2}P^{mn}\Box u\\
&+\frac{1}{4}\left(P^{mn}T^k-P^{nk}T^m\right)\nabla_ku-\frac{1}{2}P^{mk}P^{nl}\nabla^2_{k,l}u.
\end{align*}
The second evolution equation follows from \cite[Prop. 9]{Chowcross2002} and  \cite[Equ. (4)]{Chowcross2002}.
\end{proof}
\begin{lemma}
\begin{align*}
\frac{1}{2}\Box P^{ij}\nabla_iu\nabla_ju=&\frac{1}{2}\nabla_kP^{mi}\nabla_mP^{kj}\nabla_iu\nabla_ju-\frac{1}{2}T^m{{R_{km}}^k}_nP^{ni}\nabla_iu-\frac{1}{2}T^m{{R_{km}}^i}_nP^{kn}\nabla_iu\\
&+\frac{1}{4}T^i\Box u\nabla_iu-\frac{1}{4}T^iT^j\nabla^2_{i,j}u+\frac{1}{2}\nabla_k D^{kij}\nabla_iu\nabla_ju.
\end{align*}
\end{lemma}
\begin{proof}
\begin{align*}
\Box P^{ij}\nabla_iu\nabla_ju=&\nabla_kT^{kij}\nabla_iu\nabla_ju\\
=&\nabla_kT^{ikj}\nabla_iu\nabla_ju+\nabla_k D^{kij}\nabla_iu\nabla_ju+\frac{1}{2}\nabla_k\left(P^{ij}T^k-P^{jk}T^i\right)\nabla_iu\nabla_ju\\
=&\nabla_k(P^{im}\nabla_mP^{kj})\nabla_iu\nabla_ju+\nabla_k D^{kij}\nabla_iu\nabla_ju+\frac{1}{2}T^i\Box u\nabla_iu-\frac{1}{2}T^iT^j\nabla^2_{i,j}u\\
=&\nabla_kP^{im}\nabla_mP^{kj}\nabla_iu\nabla_ju+T^m\nabla^2_{k,m}P^{ki}\nabla_iu+\frac{1}{2}T^i\Box u\nabla_iu\\
&-\frac{1}{2}T^iT^j\nabla^2_{i,j}u+\nabla_k D^{kij}\nabla_iu\nabla_ju\\
=&\nabla_kP^{im}\nabla_mP^{kj}\nabla_iu\nabla_ju-T^m{{R_{km}}^k}_nP^{ni}\nabla_iu-T^m{{R_{km}}^i}_nP^{kn}\nabla_iu\\
&+\frac{1}{2}T^i\Box u\nabla_iu-\frac{1}{2}T^iT^j\nabla^2_{i,j}u+\nabla_k D^{kij}\nabla_iu\nabla_ju.
\end{align*}
\end{proof}
\begin{lemma}
\begin{align*}
\frac{1}{2}\Box\left(P^{ij}\nabla_i u\nabla_ju\right)=&\frac{1}{2}\Box P^{ij}\nabla_i u\nabla_ju+2T^{ijk}\nabla^2_{i,j}u\nabla_ku\\
&+T^i\Box\nabla_i u+P^{kl}P^{ij}\nabla^2_{l,i}u\nabla^2_{k,j}u.
\end{align*}
\end{lemma}
\begin{proof}
\begin{align*}
\Box\left(P^{ij}\nabla_i u\nabla_ju\right)=&P^{kl}\nabla_k\left(\nabla_lP^{ij}\nabla_iu\nabla_ju+2P^{ij}\nabla^2_{l,i}u\nabla_ju\right)\\
=&\Box P^{ij}\nabla_i u\nabla_ju+2T^{kij}\nabla^2_{k,i}u\nabla_ju\\
&+2T^{lij}\nabla^2_{l,i}u\nabla_ju+2T^i\Box\nabla_i u+2P^{kl}P^{ij}\nabla^2_{l,i}u\nabla^2_{k,j}u.
\end{align*}
\end{proof}
\begin{lemma}
\begin{align*}
(\partial_t \Box-\Box\partial_t) u=
&-\det P\Delta u-H\Box u+\frac{1}{4}T^i\nabla_iu\Box u-\frac{1}{4}T^iT^j\nabla^2_{i,j}u\\
&+\frac{1}{2}(\Box u)^2-\frac{1}{2}P^{ik}P^{jl}\nabla^2_{k,l}u\nabla^2_{i,j}u\\
&+\left(\nabla_kD^{knm}+\frac{1}{2}D^{kmn}\nabla_ku\right)\nabla^2_{m,n}u.
\end{align*}
\end{lemma}
\begin{proof}
\begin{align*}
(\partial_t \Box-\Box \partial_t)u
=&\partial_tP^{ij}\nabla^2_{i,j}u-P^{ij}\partial_t\Gamma_{ij}^k\nabla_ku.
\end{align*}
A calculation shows that the time derivative of the Christoffel symbols is given by
\begin{align*}
\partial_t \Gamma_{ij}^k&=g^{kl}\left(\nabla_ih_{jl}+\nabla_jh_{il}-\nabla_{l}h_{ij}\right).
\end{align*}
Thus using identity (\ref{equ: important}) we calculate
\begin{align*}
P^{ij}\partial_t \Gamma_{ij}^k=&g^{kl}\left(P^{ij}\nabla_ih_{jl}+P^{ij}\nabla_jh_{il}-P^{ij}\nabla_{l}h_{ij}\right)\\
=&g^{kl}\left(\frac{1}{2}P^{ij}\nabla_lh_{ij}+\frac{1}{2}P^{ij}\nabla_lh_{ij}-P^{ij}\nabla_{l}h_{ij}\right)=0.
\end{align*}
Hence the claim follows from Lemma \ref{ev u}.
\end{proof}
\begin{lemma}
\begin{align*}
-2\partial_tH&=2\|h\|^2+2H^2-\frac{1}{2}HT^i\nabla_iu
\\&-\frac{1}{2}\det P |\nabla u|^2
-H\Box u-\det P\Delta u\\
&+2\left(\nabla_kD^{knm}+\frac{1}{2}D^{kmn}\nabla_ku\right)h_m^lV_{ln}
-H\left|D^{ijk}\right|_V^2.
\end{align*}
\end{lemma}
\begin{proof}
\begin{align*}
\partial_tH=&\partial_t(V\det P)\\
=&\det P\partial_t V+H\partial_tu\\
=&\det P(-2h^{ij}V_{ij}-g^{ij}V_{im}V_{jn}\partial_tP^{mn})+H\partial_tu\\
=&-2\|h\|^2+H\partial_tu-h_m^lV_{ln}\partial_tP^{mn}\\
=&-2\|h\|^2+H\left(\Box u-2H+\frac{1}{2}\left(\left|D^{ijk}\right|^2_V+\left|T^i\right|_V^2\right)\right)\\
&-\left(\nabla_kD^{knm}+\frac{1}{2}D^{kmn}\nabla_ku\right)h_m^lV_{ln}+\|h\|^2+H^2\\
&-\frac{1}{4}HT^k\nabla_ku+\frac{1}{4}T^mT^nh_m^lV_{ln}-\frac{1}{2}H\Box u+\frac{1}{2}\det P\Delta u.
\end{align*}
To complete the proof, note that
\[T^iT^jh_i^lV_{lj}=\det P|\nabla u|^2.\]
\end{proof}
\begin{lemma}
\[\Box\left|D^{ijk}\right|^2_V=\cdots\]
\end{lemma}
\begin{lemma}
\[\partial_t\left|D^{ijk}\right|^2_V=\Box\left|D^{ijk}\right|^2_V+\cdots\]
\end{lemma}
\section{The computation}
Let us define
\begin{align*}
w_{kl}:=&\nabla^2_{k,l}u+V_{lm}\nabla_kP^{mn}\nabla_nu-2h_k^mV_{ml},\\
w:=&\Box u-2H=\partial_tu-\frac{1}{2}\left(\left|D^{ijk}\right|^2_V+\left|T^i\right|_V^2\right).
\end{align*}
Note that we have
\begin{align*}
(\partial_t-\Box)w
&= \frac{1}{2}\Box\left|D^{ijk}\right|^2_V+\frac{1}{2}\Box\left|T^i\right|_V^2+(\partial_t \Box-\Box\partial_t) u-2\partial_tH.
\end{align*}
\begin{lemma}
\begin{align*}
(\partial_t-\Box)w&=P^{ij}\nabla_i u\nabla_j w+ \frac{1}{2}\left(P^{ik}P^{jl}w_{ij}w_{kl}+w^2\right)+f(D^{ijk}).\\
\end{align*}
\end{lemma}
\begin{proof}
We start by multiplying
\begin{align*}
w_{ij}&=\nabla^2_{i,j}u+V_{jr}\nabla_iP^{rs}\nabla_su-2h_i^rV_{rj},\\
w_{kl}&=\nabla^2_{k,l}u+V_{lm}\nabla_kP^{mn}\nabla_nu-2h_k^mV_{ml}.
\end{align*}
We have
\begin{align*}
P^{ik}P^{jl}w_{ij}w_{kl}=&P^{ik}P^{jl}\nabla^2_{i,j}u\nabla^2_{k,l}u+2P^{ik}P^{jl}V_{lm}\nabla_kP^{mn}\nabla_nu\nabla^2_{i,j}u\\
&-4P^{ik}P^{jl}h_k^mV_{ml}\nabla^2_{i,j}u+P^{ik}P^{jl}V_{jr}V_{lm}\nabla_iP^{rs}\nabla_kP^{mn}\nabla_su\nabla_nu\\
&-4P^{ik}P^{jl}V_{jr}h_k^mV_{ml}\nabla_iP^{rs}\nabla_su+4P^{ik}P^{jl}h_i^rV_{rj}h_k^mV_{ml}\\
=&P^{ik}P^{jl}\nabla^2_{i,j}u\nabla^2_{k,l}u+2P^{ik}\nabla_kP^{nj}\nabla_nu\nabla^2_{i,j}u-4\det P g^{ij}\nabla^2_{i,j}u\\
&+P^{ik}V_{jr}\nabla_iP^{rs}\nabla_kP^{jn}\nabla_su\nabla_nu-4h^i_r\nabla_iP^{rs}\nabla_su+4\|h\|^2\\
=&P^{ik}P^{jl}\nabla^2_{i,j}u\nabla^2_{k,l}u+ {2T^{ijk}\nabla^2_{i,j}u\nabla_ku}-4\det P \Delta u\\
&+T^{ijk}V_{jr}\nabla_iP^{rs}\nabla_su\nabla_ku-4h_{j}^i\nabla_iP^{jk}\nabla_ku+ {4\|h\|^2}\\
=&P^{ik}P^{jl}\nabla^2_{i,j}u\nabla^2_{k,l}u+ {2T^{ijk}\nabla^2_{i,j}u\nabla_ku}-4\det P \Delta u-4h_{j}^i\nabla_iP^{jk}\nabla_ku\\
&+ {4\|h\|^2}+T^{jik}V_{jr}\nabla_iP^{rs}\nabla_su\nabla_ku+D^{ijk}V_{jr}\nabla_iP^{rs}\nabla_su\nabla_ku\\
&+\frac{1}{2}\left(P^{jk}T^i-P^{ik}T^j\right)V_{jr}\nabla_iP^{rs}\nabla_su\nabla_ku.
\end{align*}
Therefore,
\begin{align*}
\frac{1}{2}P^{ik}P^{jl}w_{ij}w_{kl}=&\frac{1}{2}P^{ik}P^{jl}\nabla^2_{i,j}u\nabla^2_{k,l}u+ {T^{ijk}\nabla^2_{i,j}u\nabla_ku}-2\det P \Delta u-2h_{j}^i\nabla_iP^{jk}\nabla_ku\\
&+ {2\|h\|^2}+\frac{1}{2}\nabla_mP^{ik}\nabla_iP^{ms}\nabla_su\nabla_ku+\frac{1}{2}D^{ijk}V_{jr}\nabla_iP^{rs}\nabla_su\nabla_ku.
\end{align*}
We also have
\begin{align*}
\frac{1}{2}w^2=& {\frac{1}{2}(\Box u)^2-2H\Box u}+ {2H^2}.
\end{align*}
Next we calculate
\begin{align*}
\nabla_k w=&\nabla_k\Box u-2g^{ij}\nabla_k(V_{ij}\det P)\\
=&\nabla_k(P^{ij}\nabla^2_{i,j} u)+2h_i^mV_{mj}\nabla_kP^{ij}-2H\nabla_ku\\
=&\nabla_kP^{ij}\nabla^2_{i,j}u+2h_k^i\nabla_iu+\Box\nabla_ku+2h_i^mV_{mj}\nabla_kP^{ij}-2H\nabla_ku.
\end{align*}
Therefore, we obtain
\begin{align*}
P^{kl}\nabla_k w\nabla_lu
=&P^{kl}\nabla_kP^{ij}\nabla_lu\nabla^2_{i,j}u+2\det P|\nabla u|^2+P^{kl}\Box\nabla_ku\nabla_lu\\
&+2h_i^mV_{mj}P^{kl}\nabla_kP^{ij}\nabla_lu-2HP^{kl}\nabla_ku\nabla_lu\\
=&T^{lij}\nabla^2_{i,j}u\nabla_lu+2\det P|\nabla u|^2+T^{k}\Box\nabla_ku\\
&+2h_i^mV_{mj}T^{lij}\nabla_lu-2HT^{k}\nabla_ku\\
=& {T^{ilj}\nabla^2_{i,j}u\nabla_lu}+\left(D^{lij}+\frac{1}{2}\left(P^{ij}T^l-P^{lj}T^i\right)\right)\nabla^2_{i,j}u\nabla_lu\\
&+2\det P|\nabla u|^2+ {T^{k}\Box\nabla_ku}+2h_i^mV_{mj}T^{lij}\nabla_lu-2HT^{k}\nabla_ku\\
=& {T^{ijl}\nabla^2_{i,j}u\nabla_lu}+\frac{1}{2}T^l\Box u\nabla_lu-\frac{1}{2}T^iT^j\nabla^2_{i,j}u+2\det P|\nabla u|^2\\
&+{T^{k}\Box\nabla_ku}+2h_i^kV_{kj}T^{lij}\nabla_lu-2HT^{k}\nabla_ku+D^{lij}\nabla^2_{i,j}u\nabla_lu\\
=&{T^{ijl}\nabla^2_{i,j}u\nabla_lu}+\frac{1}{2}T^l\Box u\nabla_lu-\frac{1}{2}T^iT^j\nabla^2_{i,j}u+2\det P|\nabla u|^2\\
&+{T^{k}\Box\nabla_ku}+2h_i^kV_{kj}T^{ijl}\nabla_lu-2HT^{k}\nabla_ku+D^{lij}\nabla^2_{i,j}u\nabla_lu\\
&+2h_i^kV_{kj}D^{lij}\nabla_lu+h_i^kV_{kj}(P^{ij}T^l-P^{jl}T^i)\nabla_lu\\
=&{T^{ijk}\nabla^2_{i,j}u\nabla_ku}+\frac{1}{2}T^i\Box u\nabla_iu-\frac{1}{2}T^iT^j\nabla^2_{i,j}u+\det P|\nabla u|^2\\
&+{T^i\Box\nabla_iu}+2h_{j}^i\nabla_iP^{jk}\nabla_ku-HT^{i}\nabla_iu+\left(2h_i^kV_{kj}+\nabla^2_{i,j}u\right)D^{lij}\nabla_lu.
\end{align*}
\end{proof}

\bibliographystyle{amsplain}
\begin{thebibliography}{10}
\bibitem{Andrewsharnack} B. Andrews. ``Harnack inequalities for evolving hypersurfaces." Mathematische Zeitschrift 217(1994): 179--197.
\bibitem{Chowcross2002} B. Chow and R.S. Hamilton. ``The cross curvature flow of 3-manifolds with negative sectional curvature." Turkish Journal of Mathematics 28(2004): 1--10.
\bibitem{BIS4} P. Bryan, M.N. Ivaki, J. Scheuer. ``Harnack inequalities for curvature flows in Riemannian and Lorentzian manifolds." Preprint (2016).
\end{thebibliography}
\end{document}

\begin{align*}
\nabla^2_{i,j}u+V_{ac}V_{bd}\nabla_iP^{cd}\nabla_jP^{ab}=& V_{ab}\nabla^2_{i,j}P^{ab},\\
\nabla_iu&\leftrightarrow V_{ab}\nabla_iP^{ab}.
\end{align*}

\begin{align*}
\nabla_s z=&\nabla_s\left(\left(P^{kl}\nabla_l P^{ij}-P^{jl}\nabla_lP^{ik}\right)V_{im}V_{jn}\nabla_kP^{mn}\right)\\
=&V_{im}V_{jn}\nabla_kP^{mn}\nabla_s P^{kl}\nabla_l P^{ij}\\
&+P^{kl}V_{im}V_{jn}\nabla_kP^{mn}\nabla^2_{s,l} P^{ij}\\
&-V_{im}V_{jn}\nabla_kP^{mn}\nabla_sP^{jl}\nabla_lP^{ik}\\
&-P^{jl}V_{im}V_{jn}\nabla_kP^{mn}\nabla^2_{s,l}P^{ik}\\
&-V_{ip}V_{mq}V_{jn}P^{kl}\nabla_sP^{pq}\nabla_l P^{ij}\nabla_kP^{mn}\\
&-V_{im}V_{jp}V_{nq}P^{kl}\nabla_sP^{pq}\nabla_l P^{ij}\nabla_kP^{mn}\\
&+V_{im}V_{jn}P^{kl}\nabla_l P^{ij}\nabla^2_{s,k}P^{mn}\\
&+V_{ip}V_{mq}V_{jn}P^{jl}\nabla_sP^{pq}\nabla_lP^{ik}\nabla_kP^{mn}\\
&+V_{im}V_{jp}V_{nq}P^{jl}\nabla_sP^{pq}\nabla_lP^{ik}\nabla_kP^{mn}\\
&-V_{im}V_{jn}P^{jl}\nabla_lP^{ik}\nabla^2_{s,k}P^{mn}\\
=&V_{im}V_{jn}\nabla_kP^{mn}\nabla_s P^{kl}\nabla_l P^{ij}\\
&+P^{kl}V_{im}V_{jn}\nabla_kP^{mn}\nabla^2_{s,l} P^{ij}\\
&-V_{im}V_{jn}\nabla_kP^{mn}\nabla_sP^{jl}\nabla_lP^{ik}\\
&-V_{im}\nabla_kP^{ml}\nabla^2_{s,l}P^{ik}\\
&-V_{ip}V_{mq}V_{jn}P^{kl}\nabla_sP^{pq}\nabla_l P^{ij}\nabla_kP^{mn}\\
&-V_{im}V_{jp}V_{nq}P^{kl}\nabla_sP^{pq}\nabla_l P^{ij}\nabla_kP^{mn}\\
&+V_{im}V_{jn}P^{kl}\nabla_l P^{ij}\nabla^2_{s,k}P^{mn}\\
&+V_{ip}V_{mq}\nabla_sP^{pq}\nabla_lP^{ik}\nabla_kP^{ml}\\
&+V_{im}V_{nq}\nabla_sP^{lq}\nabla_lP^{ik}\nabla_kP^{mn}\\
&-V_{im}\nabla_lP^{ik}\nabla^2_{s,k}P^{ml}.
\end{align*}
Therefore
\begin{align*}
&\nabla^2_{r,s} z\\
=&-V_{ia}V_{mb}V_{jn}\nabla_r P^{ab}\nabla_kP^{mn}\nabla_s P^{kl}\nabla_l P^{ij}-V_{ja}V_{nb}V_{im}\nabla_r P^{ab}\nabla_kP^{mn}\nabla_s P^{kl}\nabla_l P^{ij}\\
&+V_{im}V_{jn}\nabla^2_{r,k}P^{mn}\nabla_s P^{kl}\nabla_l P^{ij}+V_{im}V_{jn}\nabla_kP^{mn}\nabla^2_{r,s} P^{kl}\nabla_l P^{ij}\\
&+V_{im}V_{jn}\nabla_kP^{mn}\nabla_s P^{kl}\nabla^2_{r,l} P^{ij}-V_{ia}V_{mb}V_{jn}P^{kl}\nabla_rP^{ab}\nabla_kP^{mn}\nabla^2_{s,l} P^{ij}\\
&-V_{ja}V_{nb}V_{im}P^{kl}\nabla_rP^{ab}\nabla_kP^{mn}\nabla^2_{s,l} P^{ij}+V_{im}V_{jn}\nabla_rP^{kl}\nabla_kP^{mn}\nabla^2_{s,l} P^{ij}\\
&+V_{im}V_{jn}P^{kl}\nabla^2_{r,k}P^{mn}\nabla^2_{s,l} P^{ij}+V_{im}V_{jn}P^{kl}\nabla_kP^{mn}\nabla_r\nabla^2_{s,l} P^{ij}\\
&+V_{ia}V_{mb}V_{jn}\nabla_rP^{ab}\nabla_kP^{mn}\nabla_sP^{jl}\nabla_lP^{ik}\\
&+V_{ja}V_{nb}V_{im}\nabla_rP^{ab}\nabla_kP^{mn}\nabla_sP^{jl}\nabla_lP^{ik}-V_{im}V_{jn}\nabla^2_{r,k}P^{mn}\nabla_sP^{jl}\nabla_lP^{ik}\\
&-V_{im}V_{jn}\nabla_kP^{mn}\nabla^2_{r,s}P^{jl}\nabla_lP^{ik}-V_{im}V_{jn}\nabla_kP^{mn}\nabla_sP^{jl}\nabla^2_{r,l}P^{ik}\\
&+V_{ia}V_{mb}\nabla_rP^{ab}\nabla_kP^{ml}\nabla^2_{s,l}P^{ik}-V_{im}\nabla^2_{r,k}P^{ml}\nabla^2_{s,l}P^{ik}\\
&-V_{im}\nabla_kP^{ml}\nabla_r\nabla^2_{s,l}P^{ik}+V_{ia}V_{pb}V_{mq}V_{jn}P^{kl}\nabla_rP^{ab}\nabla_sP^{pq}\nabla_l P^{ij}\nabla_kP^{mn}\\
&+V_{ma}V_{qb}V_{ip}V_{jn}P^{kl}\nabla_rP^{ab}\nabla_sP^{pq}\nabla_l P^{ij}\nabla_kP^{mn}+V_{ja}V_{nb}V_{mq}V_{ip}P^{kl}\nabla_rP^{ab}\nabla_sP^{pq}\nabla_l P^{ij}\nabla_kP^{mn}\\
&-V_{ip}V_{mq}V_{jn}\nabla_rP^{kl}\nabla_sP^{pq}\nabla_l P^{ij}\nabla_kP^{mn}-V_{ip}V_{mq}V_{jn}P^{kl}\nabla^2_{r,s}P^{pq}\nabla_l P^{ij}\nabla_kP^{mn}\\
&-V_{ip}V_{mq}V_{jn}P^{kl}\nabla_sP^{pq}\nabla^2_{r,l} P^{ij}\nabla_kP^{mn}-V_{ip}V_{mq}V_{jn}P^{kl}\nabla_sP^{pq}\nabla_l P^{ij}\nabla^2_{r,k}P^{mn}\\
&+V_{ia}V_{mb}V_{jp}V_{nq}P^{kl}\nabla_rP^{ab}\nabla_sP^{pq}\nabla_l P^{ij}\nabla_kP^{mn}+V_{ja}V_{pb}V_{im}V_{nq}P^{kl}\nabla_rP^{ab}\nabla_sP^{pq}\nabla_l P^{ij}\nabla_kP^{mn}\\
&+V_{na}V_{qb}V_{im}V_{jp}P^{kl}\nabla_rP^{ab}\nabla_sP^{pq}\nabla_l P^{ij}\nabla_kP^{mn}-V_{im}V_{jp}V_{nq}\nabla_rP^{kl}\nabla_sP^{pq}\nabla_l P^{ij}\nabla_kP^{mn}\\
&-V_{im}V_{jp}V_{nq}P^{kl}\nabla^2_{r,s}P^{pq}\nabla_l P^{ij}\nabla_kP^{mn}-V_{im}V_{jp}V_{nq}P^{kl}\nabla_sP^{pq}\nabla^2_{r,l} P^{ij}\nabla_kP^{mn}\\
&-V_{im}V_{jp}V_{nq}P^{kl}\nabla_sP^{pq}\nabla_l P^{ij}\nabla^2_{r,k}P^{mn}-V_{ia}V_{mb}V_{jn}P^{kl}\nabla_rP^{ab}\nabla_l P^{ij}\nabla^2_{s,k}P^{mn}\\
&-V_{ja}V_{nb}V_{im}P^{kl}\nabla_rP^{ab}\nabla_l P^{ij}\nabla^2_{s,k}P^{mn}+V_{im}V_{jn}P^{kl}\nabla^2_{r,l} P^{ij}\nabla^2_{s,k}P^{mn}\\
&+V_{im}V_{jn}P^{kl}\nabla_l P^{ij}\nabla_r\nabla^2_{s,k}P^{mn}-V_{ia}V_{pb}V_{mq}\nabla_rP^{ab}\nabla_sP^{pq}\nabla_lP^{ik}\nabla_kP^{ml}\\
&-V_{ip}V_{ma}V_{mb}\nabla_rP^{ab}\nabla_sP^{pq}\nabla_lP^{ik}\nabla_kP^{ml}+V_{ip}V_{mq}\nabla^2_{r,s}P^{pq}\nabla_lP^{ik}\nabla_kP^{ml}\\
&+V_{ip}V_{mq}\nabla_sP^{pq}\nabla^2_{r,l}P^{ik}\nabla_kP^{ml}+V_{ip}V_{mq}\nabla_sP^{pq}\nabla_lP^{ik}\nabla^2_{r,k}P^{ml}\\
&-V_{ia}V_{mb}V_{nq}\nabla_rP^{ab}\nabla_sP^{lq}\nabla_lP^{ik}\nabla_kP^{mn}-V_{na}V_{qb}V_{im}\nabla_rP^{ab}\nabla_sP^{lq}\nabla_lP^{ik}\nabla_kP^{mn}\\
&+V_{im}V_{nq}\nabla^2_{r,s}P^{lq}\nabla_lP^{ik}\nabla_kP^{mn}+V_{im}V_{nq}\nabla_sP^{lq}\nabla^2_{r,l}P^{ik}\nabla_kP^{mn}+V_{im}V_{nq}\nabla_sP^{lq}\nabla_lP^{ik}\nabla^2_{r,k}P^{mn}\\
&+V_{ia}V_{mb}\nabla_rP^{ab}\nabla_lP^{ik}\nabla^2_{s,k}P^{ml}-V_{im}\nabla^2_{r,l}P^{ik}\nabla^2_{s,k}P^{ml}-V_{im}\nabla_lP^{ik}\nabla_r\nabla^2_{s,k}P^{ml}.
\end{align*} 