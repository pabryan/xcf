\documentclass{amsart}
\usepackage[ocgcolorlinks,linktoc=all]{hyperref}
\usepackage{cancel}
\hypersetup{citecolor=blue,linkcolor=red}
\newtheorem{theorem}{Theorem}
\newtheorem*{thmA}{Theorem}
\newtheorem*{thmB}{Theorem}
\newtheorem*{rem}{Remark}
\newtheorem*{thmmain}{Theorem}
\newtheorem{lemma}[theorem]{Lemma}
\newtheorem{proposition}[theorem]{Proposition}
\newtheorem*{propmain}{Proposition}
\newtheorem{corollary}[theorem]{Corollary}
\theoremstyle{definition}
\newtheorem{definition}[theorem]{Definition}
\newtheorem{example}[theorem]{Example}
\newtheorem{xca}[theorem]{Exercise}

\theoremstyle{remark}
\newtheorem{remark}[theorem]{Remark}

\newcommand{\abs}[1]{\lvert#1\rvert}
\numberwithin{equation}{section}

\newcommand{\blankbox}[2]{%
  \parbox{\columnwidth}{\centering
    \setlength{\fboxsep}{0pt}%
    \fbox{\raisebox{0pt}[#2]{\hspace{#1}}}%
  }%
}

\begin{document}

\title[Cross curvature flow]
 {Cross curvature flow}

\curraddr{}
\email{}
\date{\today}

\dedicatory{}
\subjclass[2010]{}
\keywords{}

\begin{abstract}
We prove a differential Harnack inequality for cross curvature flow. As an application, we prove that as long as the solution exists, the evolving metric has negative sectional curvature.
\end{abstract}

\maketitle

\section{Set up}
The Riemannian metric is $g_{ij},$ its inverse is $g^{ij}$. The Levi-Civita connection is given by the Christoffel symbols
\begin{equation}
\Gamma_{ij}^k=\frac{1}{2}g^{kl}\left(\partial_ig_{jl}+\partial_jg_{il}-\partial_lg_{ij}\right),
\end{equation}
and the Riemannian curvature tensor is
\begin{equation}
R_{ijk}^l=\partial_i\Gamma_{jk}^l-\partial_j\Gamma_{ik}^l+\Gamma_{jk}^p\Gamma_{ip}^l-\Gamma_{ik}^p\Gamma_{jp}^l.
\end{equation}
We lower the index to the third position, so that
\begin{equation}
R_{ijkl}=g_{kp}R_{ijl}^p.
\end{equation}
The Ricci curvature is the contraction
\begin{equation}
R_{ij}=g^{kl}R_{ikjl}.
\end{equation}
The scalar curvature $R=g^{ij}R_{ij}.$

The evolution equation of the metric is given by
\begin{equation}
\partial_tg_{ij}=2h_{ij}
\end{equation}
where
\begin{equation}
h_{ij}:=-\frac{1}{2}P^{kl}R_{ikjl}
\end{equation}
and
\begin{equation}
P_{ij}:=R_{ij}-\frac{1}{2}Rg_{ij},\quad
P^{mn}=\left(R_{ij}-\frac{1}{2}Rg_{ij}\right)g^{im}g^{jn}.
\end{equation}
We also write
\begin{equation}
P=g_{ij}P^{ij},\quad V=g^{ij}V_{ij},\quad H=g^{ij}h_{ij},\quad  T^{kij}=P^{kl}\nabla_l P^{ij}.
\end{equation}
Let $V_{ij}$ be the inverse of $P^{ij}$. We mention that
\begin{equation}
h_{ij}=V_{ij}\det P,
\end{equation}
where $\det P = \det (g_{km} P^{ml}).$

We recall the following identities from \cite[Lemma 1]{Chowcross2002}:
\begin{align}\label{equ: important}
P^{ij}\nabla_ih_{jk}=\frac{1}{2}P^{ij}\nabla_kh_{ij},\quad\nabla_iP^{ij}=0.
\end{align}
For simplicity we set
\[u:=\log\det P.\]
In the sequel, $\mu_{ijk}$ is the volume form and is nonzero only when $i,j,k$ are distinct, in which case it is sign of the permutation $(ijk)$. Using the volume form, we may express $P^{mn}$ as
\begin{equation}
P^{mn}=-\frac{1}{4}\mu^{ijm}\mu^{kln}R_{ijkl}.
\end{equation}
\section{Harnack quadratic}
A solution to cross curvature flow is a cross curvature soliton if there exists a vector field $V$ and $\lambda\in \mathbb{R}$ such that at some time
\begin{equation}\label{soliton 0}
2h_{ij}+\nabla_iV_j+\nabla_jV_i=2\lambda g_{ij}.
\end{equation}
For an expanding soliton, (\ref{soliton 0}) holds with $\lambda=\frac{1}{4t}.$
To obtain a suitable Harnack quadratic for XCF, we follow Hamilton's procedure in obtaining his trace Harnack quantity for the Ricci flow.

Consider gradient solitons which are by definition, when $V_i=\nabla_if$ for a smooth function $f$\footnote{Gradient solitons only serves as an inspiration, the proof of Harnack inequality does not have anything to do with the existence of gradient solitons.}. Then the soliton equation reads
\begin{equation}\label{eq: soliton}
h_{ij}+\nabla_iV_j=\lambda g_{ij}.
\end{equation}
We take the covariant derivative of both sides with respect to $\nabla_l:$
\begin{align}\label{eq: soliton derv}
\nabla_lh_{ij}+\nabla_l\nabla_iV_j=0.
\end{align}
We also take trace of (\ref{eq: soliton}) with respect to $P^{ij}$:
\begin{align*}
3\det P+P^{ij}\nabla_iV_j=\lambda P.
\end{align*}
Taking the covariant derivative of both sides with respect to $\nabla_l$ yields
\begin{align*}
3\nabla_l\det P+\nabla_l P^{ij}\nabla_iV_j+P^{ij}\nabla_l\nabla_iV_j=\lambda \nabla_l P.
\end{align*}
Since
\begin{align}\label{eq: soliton derv 2}
\nabla_l\nabla_iV_j-\nabla_i\nabla_lV_j={R_{lij}}^mV_m,
\end{align}
we obtain
\begin{align*}
\lambda \nabla_l P&=3\nabla_l\det P+\nabla_l P^{ij}\nabla_iV_j+P^{ij}(\nabla_i\nabla_lV_j+{R_{lij}}^mV_m)\\
&=3\nabla_l\det P+\nabla_l P^{ij}\nabla_iV_j+P^{ij}\nabla_i\nabla_lV_j+2h_l^mV_m.
\end{align*}
Putting (\ref{eq: soliton}), (\ref{eq: soliton derv}) together we obtain
\begin{align}\label{eq:1}
\lambda \nabla_l P
=&3\nabla_l\det P+2h_l^mV_m+\nabla_l P^{ij}(\lambda g_{ij}-h_{ij})-P^{ij}\nabla_ih_{jl}\\
=&3\nabla_l\det P+2h_l^mV_m+\lambda \nabla_l P-h_{ij}\nabla_l P^{ij}-\frac{1}{2}P^{ij}\nabla_lh_{ij}.\nonumber
\end{align}
We may rewrite (\ref{eq:1}) as
\[3\nabla_l\det P+2h_l^mV_m-h_{ij}\nabla_l P^{ij}-\frac{1}{2}P^{ij}\nabla_lh_{ij}=0.\]
We divide both sides by $\det P:$
\begin{equation}\label{eq:a}
\nabla_lu+\frac{2}{\det P}h_l^mV_m=0,
\end{equation}
where we used
\begin{equation*}
\frac{P^{ij}}{\det P}\nabla_lh_{ij}=2\nabla_lu,\quad V_{ij}\nabla_lP^{ij}=\nabla_lu.
\end{equation*}
Taking the covariant derivative of (\ref{eq:a}) with respect to $\nabla_k$ yields
\begin{equation*}
\nabla_k\nabla_lu+\frac{2}{\det P}\nabla_kh_l^mV_m-\frac{2}{\det P}h_l^mV_m\nabla_ku+2V_l^mh_{km}-2\lambda V_{kl}=0.
\end{equation*}
Thus using (\ref{eq:a}) we obtain
\begin{equation*}
\nabla_k\nabla_lu+\nabla_ku\nabla_lu+\frac{2}{\det P}\nabla_kh_l^mV_m+2V_l^mh_{km}-2\lambda V_{kl}=0.
\end{equation*}
We take the trace with respect to $P^{kl}:$
\begin{equation}\label{eq:4}
\Box u+P^{kl}\nabla_ku\nabla_lu+2\nabla_kuV^k+2H-6\lambda=0.
\end{equation}
On the other hand, there holds
\begin{align}\label{first derv test}
\nabla^mh_{ij}-\nabla_jh_i^m={{{R_j}^{m}}_{i}}^nV_n\Rightarrow 4V_{kl}V^kV^l+2\nabla_kuV^k=0.
\end{align}
Hence adding (\ref{first derv test}) to (\ref{eq:4}) implies that
\begin{align*}
\Box u+4V_{kl}V^kV^l+4\nabla_kuV^k+P^{kl} \nabla_ku\nabla_lu
-2H+6\lambda=0.
\end{align*}
Note that for any vector $W$:
\[4\nabla_k uW^k+4 V_{kl}W^kW^l+ P^{kl}\nabla_ku\nabla_lu\geq 0.\]
Thus Lemma \ref{ev u} proposes the following candidate as the  Harnack inequality:
 \begin{align}\label{harnack est}
\partial_tu-\left(P^{jl}\nabla_lP^{ik}-P^{kl}\nabla_l P^{ij}\right)\nabla_kV_{ij}+\frac{3}{2t}\geq 0.
\end{align}
In Euclidean or Minkowski space, we have
 \begin{align}\label{question}
 \left(P^{jl}\nabla_lP^{ik}-P^{kl}\nabla_l P^{ij}\right)\nabla_kV_{ij}=\frac{1}{2}P^{ij}\nabla_i u\nabla_j u.
 \end{align}
In Minkowski 4-space, inequality (\ref{harnack est}) was proven  by the authors \cite{BIS4}.
\section{Basic evolution equations}
\begin{lemma}\label{ev u}
The following evolution equations hold:
\begin{align}
\partial_tP^{ij}=&\Box P^{ij}-\nabla_lP^{ki}\nabla_kP^{jn}-g^{ij}\det P-HP^{ij},
\end{align}
\begin{align}
\partial_t u=&\Box u+\left(P^{jl}\nabla_lP^{ik}-P^{kl}\nabla_l P^{ij}\right)\nabla_kV_{ij}-2H.
\end{align}
\end{lemma}
\begin{proof}
The evolution equation of the Riemann curvature tensor is given by the standard formula
\begin{align*}
\partial_t R_{ijkl}=&\nabla_i\nabla_lh_{jk}+\nabla_j\nabla_kh_{il}-\nabla_i\nabla_kh_{jl}-\nabla_j\nabla_lh_{ik}\\
&+g^{pq}(R_{ijpk}h_{ql}+R_{ijpl}h_{qk}).
\end{align*}
On the other hand, the evolution of the volume form is given by $$\partial_t\mu_{ijk}=H\mu_{ijk},\quad\partial_t\mu^{ijk}=-H\mu^{ijk}.$$
Since $P^{mn}=-\frac{1}{4}\mu^{ijm}\mu^{kln}R_{ijkl}$, we obtain
\begin{align*}
\partial_tP^{mn}=&\mu^{ijm}\mu^{kln}\nabla_i\nabla_kh_{jl}-\frac{1}{2}\mu^{ijm}\mu^{kln}g^{pq}R_{ijpl}h_{qk}-2HP^{mn}.
\end{align*}
Now using the identity
\[\frac{1}{2}\mu^{ijm}\mu^{kln}g^{pq}R_{ijpl}h_{qk}+HP^{mn}=g^{mn}\det P,\]
we arrive at
\begin{align*}
\partial_tP^{mn}=&\nabla_k\nabla_l(P^{kl}P^{mn}-P^{km}P^{ln})-g^{mn}\det P-HP^{mn}.
\end{align*}
Next we calculate the evolution equation of $u:$
\begin{align*}
\partial_t u=& V_{ij}\left(\Box P^{ij}-\nabla_lP^{ik}\nabla_kP^{jl}-g^{ij}\det P -HP^{ij}\right)- g_{ij}\partial_t g^{ij}\\
=&V_{ij}P^{kl}\nabla_k\nabla_lP^{ij}- V_{ij}\nabla_lP^{ik}\nabla_kP^{jl}-2H\\
=&P^{kl}\nabla_k\left(V_{ij}\nabla_lP^{ij}\right)-P^{kl}\nabla_k V_{ij}\nabla_l P^{ij}-V_{ij}\nabla_lP^{ik}\nabla_kP^{jl}-2H\\
=&\Box u-P^{kl}\nabla_k V_{ij}\nabla_l P^{ij}+P^{jl}\nabla_kV_{ij}\nabla_lP^{ik}-2H.
\end{align*}
\end{proof}
\begin{lemma}
\begin{align}
(\partial_t \Box-\Box\partial_t) u=&\partial_tP^{ij}\cdot\nabla_i\nabla_ju.
\end{align}
\end{lemma}
\begin{proof}
\begin{align*}
(\partial_t \Box-\Box \partial_t)u
=&\partial_tP^{ij}\cdot\nabla_i\nabla_ju-P^{ij}\partial_t\Gamma_{ij}^k\cdot\nabla_ku.
\end{align*}
A calculation shows that the time derivative of the Christoffel symbols is give by
\begin{align}
\partial_t \Gamma_{ij}^k&=g^{kl}\left(\nabla_ih_{jl}+\nabla_jh_{il}-\nabla_{l}h_{ij}\right).
\end{align}
Thus using identity (\ref{equ: important}) we calculate
\begin{align*}
P^{ij}\partial_t \Gamma_{ij}^k=&g^{kl}\left(P^{ij}\nabla_ih_{jl}+P^{ij}\nabla_jh_{il}-P^{ij}\nabla_{l}h_{ij}\right)\\
=&g^{kl}\left(\frac{1}{2}P^{ij}\nabla_lh_{ij}+\frac{1}{2}P^{ij}\nabla_lh_{ij}-P^{ij}\nabla_{l}h_{ij}\right)=0.
\end{align*}
\end{proof}
\begin{lemma}
\begin{equation}
-2\partial_tH=4\|h\|^2+2\frac{(h^2)_{ij}}{\det P}\partial_tP^{ij}-2H\partial_tu.
\end{equation}
\end{lemma}
\begin{proof}
\begin{align*}
\partial_tH=&\partial_t(V\det P)\\
=&\det P\partial_t V+H\partial_tu\\
=&\det P(-2h^{ij}V_{ij}-g^{ij}V_{im}V_{jn}\partial_tP^{mn})+H\partial_tu\\
=&-2\|h\|^2-\frac{(h^2)_{mn}}{\det P}\partial_tP^{mn}+H\partial_tu.
\end{align*}
\end{proof}
Let us define
\begin{align*}
w_{kl}:=&\nabla_k\nabla_lu+V_{lm}\nabla_kP^{mn}\nabla_nu-2h_k^mV_{ml},\\
z:=&\left(P^{jl}\nabla_lP^{ik}-P^{kl}\nabla_l P^{ij}\right)\nabla_kV_{ij}=\left(P^{kl}\nabla_l P^{ij}-P^{jl}\nabla_lP^{ik}\right)V_{im}V_{jn}\nabla_kP^{mn},\\
w:=&P^{kl}w_{kl}=\Box u-2H=\partial_tu-z.\\
\end{align*}
Note that we have
\begin{align}
(\partial_t-\Box)w
&=\Box z-2\partial_tH+\partial_tP^{ij}\cdot\nabla_i\nabla_ju,\quad P^{ik}P^{jl}w_{ij}w_{kl}\geq \frac{1}{3}w^2.
\end{align}
\begin{lemma}
\begin{align*}
\Box z-2\partial_tH+\partial_tP^{ij}\cdot\nabla_i\nabla_ju&=P^{ij}\nabla_i u\nabla_j w+ \frac{1}{2}\left(P^{ik}P^{jl}w_{ij}w_{kl}+w^2\right).\\
\end{align*}
\end{lemma}
\begin{proof}
We start by multiplying
\begin{align*}
w_{ij}&=\nabla_i\nabla_ju+V_{jr}\nabla_iP^{rs}\nabla_su-2h_i^rV_{rj},\\
w_{kl}&=\nabla_k\nabla_lu+V_{lm}\nabla_kP^{mn}\nabla_nu-2h_k^mV_{ml}.
\end{align*}
We have
\begin{align*}
P^{ik}P^{jl}w_{ij}w_{kl}=&P^{ik}P^{jl}\nabla_i\nabla_ju\nabla_k\nabla_lu+2P^{ik}P^{jl}V_{lm}\nabla_kP^{mn}\nabla_nu\nabla_i\nabla_ju\\
&-4P^{ik}P^{jl}h_k^mV_{ml}\nabla_i\nabla_ju+P^{ik}P^{jl}V_{jr}V_{lm}\nabla_iP^{rs}\nabla_kP^{mn}\nabla_su\nabla_nu\\
&-4P^{ik}P^{jl}V_{jr}h_k^mV_{ml}\nabla_iP^{rs}\nabla_su+4P^{ik}P^{jl}h_i^rV_{rj}h_k^mV_{ml}\\
=&P^{ik}P^{jl}\nabla_i\nabla_ju\nabla_k\nabla_lu+2P^{ik}\nabla_kP^{nj}\nabla_nu\nabla_i\nabla_ju\\
&-4\det P g^{ij}\nabla_i\nabla_ju+P^{ik}V_{jr}\nabla_iP^{rs}\nabla_kP^{jn}\nabla_su\nabla_nu\\
&-4h_{ir}\nabla_iP^{rs}\nabla_su+4\|h\|^2.
\end{align*}
We also have
\begin{align*}
w^2=&(\Box u)^2-4H\Box u+4H^2.
\end{align*}
Next we calculate
\begin{align*}
\nabla_k w=&\nabla_k\Box u-2g^{ij}\nabla_k(V_{ij}\det P)\\
=&\nabla_k(P^{ij}\nabla_i\nabla_j u)+\frac{2}{\det P}(h^2)_{ij}\nabla_kP^{ij}-2H\nabla_ku\\
=&\nabla_kP^{ij}\nabla_i\nabla_ju+2h_k^i\nabla_iu+\Box\nabla_ku+\frac{2}{\det P}(h^2)_{ij}\nabla_kP^{ij}-2H\nabla_ku.
\end{align*}
Therefore, we obtain
\begin{align*}
P^{kl}\nabla_k w\nabla_lu
=&P^{kl}\nabla_kP^{ij}\nabla_lu\nabla_i\nabla_ju+2\det Pg^{ij}\nabla_iu\nabla_ju+P^{kl}\Box\nabla_ku\nabla_lu\\
&+\frac{2}{\det P}(h^2)_{ij}P^{kl}\nabla_kP^{ij}\nabla_lu-2HP^{kl}\nabla_ku\nabla_lu.
\end{align*}
In addition, we have
\begin{align*}
-2\partial_tH+\partial_tP^{ij}\cdot\nabla_i\nabla_ju=&4\|h\|^2+2\frac{(h^2)_{ij}}{\det P}\Box P^{ij}-2\frac{(h^2)_{ij}}{\det P}\nabla_lP^{ik}\nabla_kP^{jl}\\
&-2\|h\|^2-2\frac{H}{\det P}P^{ij}(h^2)_{ij}\\
&+\Box P^{ij}\nabla_i\nabla_ju-\nabla_lP^{ik}\nabla_kP^{jl}\nabla_i\nabla_ju\\
&-g^{ij}\det P\nabla_i\nabla_ju-H\Box u-2H\partial_tu\\
=&2\|h\|^2+2\frac{(h^2)_{ij}}{\det P}\Box P^{ij}-2\frac{(h^2)_{ij}}{\det P}\nabla_lP^{ik}\nabla_kP^{jl}\\
&+2H^2+\Box P^{ij}\nabla_i\nabla_ju-\nabla_lP^{ik}\nabla_kP^{jl}\nabla_i\nabla_ju\\
&-g^{ij}\det P\nabla_i\nabla_ju-3H\Box u-2Hz.
\end{align*}
On the other hand, we have
\begin{align*}
\nabla_s z=&\nabla_s\left(\left(P^{kl}\nabla_l P^{ij}-P^{jl}\nabla_lP^{ik}\right)V_{im}V_{jn}\nabla_kP^{mn}\right)\\
=&V_{im}V_{jn}\nabla_kP^{mn}\nabla_s P^{kl}\nabla_l P^{ij}\\
&+P^{kl}V_{im}V_{jn}\nabla_kP^{mn}\nabla_s\nabla_l P^{ij}\\
&-V_{im}V_{jn}\nabla_kP^{mn}\nabla_sP^{jl}\nabla_lP^{ik}\\
&-P^{jl}V_{im}V_{jn}\nabla_kP^{mn}\nabla_s\nabla_lP^{ik}\\
&-V_{ip}V_{mq}V_{jn}P^{kl}\nabla_sP^{pq}\nabla_l P^{ij}\nabla_kP^{mn}\\
&-V_{im}V_{jp}V_{nq}P^{kl}\nabla_sP^{pq}\nabla_l P^{ij}\nabla_kP^{mn}\\
&+V_{im}V_{jn}P^{kl}\nabla_l P^{ij}\nabla_s\nabla_kP^{mn}\\
&+V_{ip}V_{mq}V_{jn}P^{jl}\nabla_sP^{pq}\nabla_lP^{ik}\nabla_kP^{mn}\\
&+V_{im}V_{jp}V_{nq}P^{jl}\nabla_sP^{pq}\nabla_lP^{ik}\nabla_kP^{mn}\\
&-V_{im}V_{jn}P^{jl}\nabla_lP^{ik}\nabla_s\nabla_kP^{mn}\\
=&V_{im}V_{jn}\nabla_kP^{mn}\nabla_s P^{kl}\nabla_l P^{ij}\\
&+P^{kl}V_{im}V_{jn}\nabla_kP^{mn}\nabla_s\nabla_l P^{ij}\\
&-V_{im}V_{jn}\nabla_kP^{mn}\nabla_sP^{jl}\nabla_lP^{ik}\\
&-V_{im}\nabla_kP^{ml}\nabla_s\nabla_lP^{ik}\\
&-V_{ip}V_{mq}V_{jn}P^{kl}\nabla_sP^{pq}\nabla_l P^{ij}\nabla_kP^{mn}\\
&-V_{im}V_{jp}V_{nq}P^{kl}\nabla_sP^{pq}\nabla_l P^{ij}\nabla_kP^{mn}\\
&+V_{im}V_{jn}P^{kl}\nabla_l P^{ij}\nabla_s\nabla_kP^{mn}\\
&+V_{ip}V_{mq}\nabla_sP^{pq}\nabla_lP^{ik}\nabla_kP^{ml}\\
&+V_{im}V_{nq}\nabla_sP^{lq}\nabla_lP^{ik}\nabla_kP^{mn}\\
&-V_{im}\nabla_lP^{ik}\nabla_s\nabla_kP^{ml}.
\end{align*}
Therefore
\begin{align*}
&\nabla_r\nabla_s z\\
=&-V_{ia}V_{mb}V_{jn}\nabla_r P^{ab}\nabla_kP^{mn}\nabla_s P^{kl}\nabla_l P^{ij}-V_{ja}V_{nb}V_{im}\nabla_r P^{ab}\nabla_kP^{mn}\nabla_s P^{kl}\nabla_l P^{ij}\\
&+V_{im}V_{jn}\nabla_r\nabla_kP^{mn}\nabla_s P^{kl}\nabla_l P^{ij}+V_{im}V_{jn}\nabla_kP^{mn}\nabla_r\nabla_s P^{kl}\nabla_l P^{ij}\\
&+V_{im}V_{jn}\nabla_kP^{mn}\nabla_s P^{kl}\nabla_r\nabla_l P^{ij}-V_{ia}V_{mb}V_{jn}P^{kl}\nabla_rP^{ab}\nabla_kP^{mn}\nabla_s\nabla_l P^{ij}\\
&-V_{ja}V_{nb}V_{im}P^{kl}\nabla_rP^{ab}\nabla_kP^{mn}\nabla_s\nabla_l P^{ij}+V_{im}V_{jn}\nabla_rP^{kl}\nabla_kP^{mn}\nabla_s\nabla_l P^{ij}\\
&+V_{im}V_{jn}P^{kl}\nabla_r\nabla_kP^{mn}\nabla_s\nabla_l P^{ij}+V_{im}V_{jn}P^{kl}\nabla_kP^{mn}\nabla_r\nabla_s\nabla_l P^{ij}\\
&+V_{ia}V_{mb}V_{jn}\nabla_rP^{ab}\nabla_kP^{mn}\nabla_sP^{jl}\nabla_lP^{ik}\\
&+V_{ja}V_{nb}V_{im}\nabla_rP^{ab}\nabla_kP^{mn}\nabla_sP^{jl}\nabla_lP^{ik}-V_{im}V_{jn}\nabla_r\nabla_kP^{mn}\nabla_sP^{jl}\nabla_lP^{ik}\\
&-V_{im}V_{jn}\nabla_kP^{mn}\nabla_r\nabla_sP^{jl}\nabla_lP^{ik}-V_{im}V_{jn}\nabla_kP^{mn}\nabla_sP^{jl}\nabla_r\nabla_lP^{ik}\\
&+V_{ia}V_{mb}\nabla_rP^{ab}\nabla_kP^{ml}\nabla_s\nabla_lP^{ik}-V_{im}\nabla_r\nabla_kP^{ml}\nabla_s\nabla_lP^{ik}\\
&-V_{im}\nabla_kP^{ml}\nabla_r\nabla_s\nabla_lP^{ik}+V_{ia}V_{pb}V_{mq}V_{jn}P^{kl}\nabla_rP^{ab}\nabla_sP^{pq}\nabla_l P^{ij}\nabla_kP^{mn}\\
&+V_{ma}V_{qb}V_{ip}V_{jn}P^{kl}\nabla_rP^{ab}\nabla_sP^{pq}\nabla_l P^{ij}\nabla_kP^{mn}+V_{ja}V_{nb}V_{mq}V_{ip}P^{kl}\nabla_rP^{ab}\nabla_sP^{pq}\nabla_l P^{ij}\nabla_kP^{mn}\\
&-V_{ip}V_{mq}V_{jn}\nabla_rP^{kl}\nabla_sP^{pq}\nabla_l P^{ij}\nabla_kP^{mn}-V_{ip}V_{mq}V_{jn}P^{kl}\nabla_r\nabla_sP^{pq}\nabla_l P^{ij}\nabla_kP^{mn}\\
&-V_{ip}V_{mq}V_{jn}P^{kl}\nabla_sP^{pq}\nabla_r\nabla_l P^{ij}\nabla_kP^{mn}-V_{ip}V_{mq}V_{jn}P^{kl}\nabla_sP^{pq}\nabla_l P^{ij}\nabla_r\nabla_kP^{mn}\\
&+V_{ia}V_{mb}V_{jp}V_{nq}P^{kl}\nabla_rP^{ab}\nabla_sP^{pq}\nabla_l P^{ij}\nabla_kP^{mn}+V_{ja}V_{pb}V_{im}V_{nq}P^{kl}\nabla_rP^{ab}\nabla_sP^{pq}\nabla_l P^{ij}\nabla_kP^{mn}\\
&+V_{na}V_{qb}V_{im}V_{jp}P^{kl}\nabla_rP^{ab}\nabla_sP^{pq}\nabla_l P^{ij}\nabla_kP^{mn}-V_{im}V_{jp}V_{nq}\nabla_rP^{kl}\nabla_sP^{pq}\nabla_l P^{ij}\nabla_kP^{mn}\\
&-V_{im}V_{jp}V_{nq}P^{kl}\nabla_r\nabla_sP^{pq}\nabla_l P^{ij}\nabla_kP^{mn}-V_{im}V_{jp}V_{nq}P^{kl}\nabla_sP^{pq}\nabla_r\nabla_l P^{ij}\nabla_kP^{mn}\\
&-V_{im}V_{jp}V_{nq}P^{kl}\nabla_sP^{pq}\nabla_l P^{ij}\nabla_r\nabla_kP^{mn}-V_{ia}V_{mb}V_{jn}P^{kl}\nabla_rP^{ab}\nabla_l P^{ij}\nabla_s\nabla_kP^{mn}\\
&-V_{ja}V_{nb}V_{im}P^{kl}\nabla_rP^{ab}\nabla_l P^{ij}\nabla_s\nabla_kP^{mn}+V_{im}V_{jn}P^{kl}\nabla_r\nabla_l P^{ij}\nabla_s\nabla_kP^{mn}\\
&+V_{im}V_{jn}P^{kl}\nabla_l P^{ij}\nabla_r\nabla_s\nabla_kP^{mn}-V_{ia}V_{pb}V_{mq}\nabla_rP^{ab}\nabla_sP^{pq}\nabla_lP^{ik}\nabla_kP^{ml}\\
&-V_{ip}V_{ma}V_{mb}\nabla_rP^{ab}\nabla_sP^{pq}\nabla_lP^{ik}\nabla_kP^{ml}+V_{ip}V_{mq}\nabla_r\nabla_sP^{pq}\nabla_lP^{ik}\nabla_kP^{ml}\\
&+V_{ip}V_{mq}\nabla_sP^{pq}\nabla_r\nabla_lP^{ik}\nabla_kP^{ml}+V_{ip}V_{mq}\nabla_sP^{pq}\nabla_lP^{ik}\nabla_r\nabla_kP^{ml}\\
&-V_{ia}V_{mb}V_{nq}\nabla_rP^{ab}\nabla_sP^{lq}\nabla_lP^{ik}\nabla_kP^{mn}-V_{na}V_{qb}V_{im}\nabla_rP^{ab}\nabla_sP^{lq}\nabla_lP^{ik}\nabla_kP^{mn}\\
&+V_{im}V_{nq}\nabla_r\nabla_sP^{lq}\nabla_lP^{ik}\nabla_kP^{mn}+V_{im}V_{nq}\nabla_sP^{lq}\nabla_r\nabla_lP^{ik}\nabla_kP^{mn}+V_{im}V_{nq}\nabla_sP^{lq}\nabla_lP^{ik}\nabla_r\nabla_kP^{mn}\\
&+V_{ia}V_{mb}\nabla_rP^{ab}\nabla_lP^{ik}\nabla_s\nabla_kP^{ml}-V_{im}\nabla_r\nabla_lP^{ik}\nabla_s\nabla_kP^{ml}-V_{im}\nabla_lP^{ik}\nabla_r\nabla_s\nabla_kP^{ml}.
\end{align*}

\begin{align*}
\nabla_i\nabla_ju+V_{ac}V_{bd}\nabla_iP^{cd}\nabla_jP^{ab}=& V_{ab}\nabla_i\nabla_jP^{ab},\\
\nabla_iu&\leftrightarrow V_{ab}\nabla_iP^{ab}.
\end{align*}
\end{proof}
\begin{theorem}
Let $(M^3,g(t))$, $t\in [0,T)$, be a solution to the XCF on a closed 3-manifold starting from a metric $g(0)$ with negative sectional curvature. Then $g(t)$ has negative sectional curvature for all $t<T.$
Moreover, it is not possible that $T<\infty$ and $\inf_{M\times [0,T)}\det P=0.$
\end{theorem}
\begin{proof}
Suppose $0<t_{\ast}<T$ is the first time that $g(t_{\ast})$ has a zero sectional curvature at some point $x_{\ast}$. At $(x_{\ast},t_{\ast})$ we must have $\det P(x_{\ast},t_{\ast})=0.$
By our Harnack inequality, we know that $\det P(\cdot,t) t^{\frac{2}{3}}$ is increasing in time on $(0,t_{\ast})$; therefore, for any $t\in [t_{\ast}/2, t_{\ast})$ we have
\[\det P(\cdot,t)\geq \det P(\cdot,t_{\ast}/2)\left(\frac{1}{2}\right)^{\frac{2}{3}}>0.\]
Thus by continuity of the solution we obtain $\det P(\cdot,t_{\ast})>0$, which yields a contradiction.

By the first part of the theorem, $P^{ij}$ is invertible on $[t_1,t_2]$ for any $t_1,t_2$ with $0<t_1\leq t_2<T.$ Thus the Harnack inequality is available on $[t_1,t_2]$ and it gives
\[\det P(\cdot,t_2)\geq \det P(\cdot,t_1)\left(\frac{t_1}{t_2}\right)^{\frac{2}{3}}.\]
Allowing $t_2\to T$ completes the proof.
\end{proof}
\bibliographystyle{amsplain}
\begin{thebibliography}{10}
\bibitem{Andrewsharnack} B. Andrews. ``Harnack inequalities for evolving hypersurfaces." Mathematische Zeitschrift 217(1994): 179--197.
\bibitem{Chowcross2002} B. Chow and R.S. Hamilton. ``The cross curvature flow of 3-manifolds with negative sectional curvature." Turkish Journal of Mathematics 28(2004): 1--10.
\bibitem{BIS4} P. Bryan, M.N. Ivaki, J. Scheuer. ``Harnack inequalities for curvature flows in Riemannian and Lorentzian manifolds." Preprint (2016).
\end{thebibliography}
\end{document}

\begin{align*}
\nabla_i\nabla_ju+V_{ac}V_{bd}\nabla_iP^{cd}\nabla_jP^{ab}=& V_{ab}\nabla_i\nabla_jP^{ab},\\
\nabla_iu&\leftrightarrow V_{ab}\nabla_iP^{ab}.
\end{align*} 