\documentclass{amsart}
\usepackage[ocgcolorlinks,linktoc=all]{hyperref}
\usepackage{cancel}
\hypersetup{citecolor=blue,linkcolor=red}
\newtheorem{theorem}{Theorem}
\newtheorem*{thmA}{Theorem}
\newtheorem*{thmB}{Theorem}
\newtheorem*{rem}{Remark}
\newtheorem*{thmmain}{Theorem}
\newtheorem{lemma}[theorem]{Lemma}
\newtheorem{proposition}[theorem]{Proposition}
\newtheorem*{propmain}{Proposition}
\newtheorem{corollary}[theorem]{Corollary}
\theoremstyle{definition}
\newtheorem{definition}[theorem]{Definition}
\newtheorem{example}[theorem]{Example}
\newtheorem{xca}[theorem]{Exercise}

\theoremstyle{remark}
\newtheorem{remark}[theorem]{Remark}

\newcommand{\abs}[1]{\lvert#1\rvert}
\numberwithin{equation}{section}

\newcommand{\blankbox}[2]{%
  \parbox{\columnwidth}{\centering
    \setlength{\fboxsep}{0pt}%
    \fbox{\raisebox{0pt}[#2]{\hspace{#1}}}%
  }%
}

\begin{document}

\title[Cross curvature flow]
 {Cross curvature flow}

\curraddr{}
\email{}
\date{\today}

\dedicatory{}
\subjclass[2010]{}
\keywords{}

\begin{abstract}


\end{abstract}

\maketitle

\section{Set up}
The Riemannian metric is $g_{ij},$ its inverse is $g^{ij}$. The Levi-Civita connection is given by the Christoffel symbols


\begin{equation}
\Gamma_{ij}^k=\frac{1}{2}g^{kl}\left(\partial_ig_{jl}+\partial_jg_{il}-\partial_lg_{ij}\right),
\end{equation}
and the Riemannian curvature tensor is

\begin{equation}
R_{ijk}^l=\partial_i\Gamma_{jk}^l-\partial_j\Gamma_{ik}^l+\Gamma_{jk}^p\Gamma_{ip}^l-\Gamma_{ik}^p\Gamma_{jp}^l.
\end{equation}

We lower the index to the third position, so that
\begin{equation}
R_{ijkl}=g_{kp}R_{ijl}^p.
\end{equation}

The Ricci curvature is the contraction
\begin{equation}
R_{ij}=g^{kl}R_{ikjl}.
\end{equation}
The scalar curvature $R=g^{ij}R_{ij}.$

The evolution equation of the metric is given by

\begin{equation}
\partial_tg_{ij}=2h_{ij}
\end{equation}
where
\begin{equation}
h_{ij}:=-\frac{1}{2}P^{kl}R_{ikjl}
\end{equation}
and
\begin{equation}
P_{ij}:=(R_{ij}-\frac{1}{2}Rg_{ij}),\quad
P^{mn}=(R_{ij}-\frac{1}{2}Rg_{ij})g^{im}g^{jn}.
\end{equation}
We also write
\begin{equation}
P=g_{ij}P^{ij},\quad V=g^{ij}V_{ij},\quad H=g^{ij}h_{ij},\quad  T^{kij}=P^{kl}\nabla_l P^{ij}.
\end{equation}
We mention that
\begin{equation}
h_{ij}=V_{ij}\det P
\end{equation}
where $V_{ij}$ is the inverse of $P^{ij}.$
For simplicity we set
\[u:=\log\det p.\]
In the sequel, $\mu_{ijk}$ is the volume form and is nonzero only when $i,j,k$ are distinct, in which case it is sign of the permutation $(ijk)$. Using the volume form, we may express $P^{mn}$ as
\begin{equation}
P^{mn}=-\frac{1}{4}\mu^{ijm}\mu^{kln}R_{ijkl}.
\end{equation}

The following is the main result of the paper.
\begin{theorem}
\[\partial_t u-\frac{1}{2}P^{ij}\nabla_iu\nabla_ju+\frac{3}{2t}\geq 0. \]
\end{theorem}
Define a family of time dependent diffeomorphisms by
\begin{align*}
\varphi_t&:M^3\to M^3\\
\partial_t\varphi&=-\frac{1}{2}\operatorname{grad}_h\det P=:W.
\end{align*}
Under this re-parametrization, the metric $\bar{g}=\varphi_t^{\ast}g$ evolves by
\begin{align}
\partial_t \bar{g}_{ij}&=2(\bar{h}_{ij}+\frac{1}{2}(\mathcal{L}_{W^{\natural}}g)_{ij}).
\end{align}
For simplicity, we set
\begin{align}
v_{ij}:=\frac{1}{2}(\mathcal{L}_{W^{\natural}}g)_{ij},\quad v=g^{ij}v_{ij}=\operatorname{div}W^{\natural}\textcolor[rgb]{1.00,0.00,0.00}{=}-\frac{1}{2}\Box u.
\end{align}
In this parametrization, to prove our main theorem, it suffices to show that
\[\partial_t \bar{u}+\frac{3}{2t}\geq 0. \]
For convenience, we drop the over bar in the sequel.
\begin{lemma}
We have the following evolution equations:
\begin{align}
\partial_tP^{mn}=&\Box P^{mn}-\nabla_lP^{km}\nabla_kP^{ln}-g^{mn}\det P-HP^{mn}\\
&+\mu^{ijm}\mu^{kln}\left(\nabla^2_{ik}v_{jl}-\frac{1}{2}g^{pq}R_{ijpl}v_{qk}\right)-2vP^{mn}.
\end{align}
\begin{align}
\partial_t u=&\Box u+\left(P^{jl}\nabla_lP^{ik}-P^{kl}\nabla_l P^{ij}\right)\nabla_kV_{ij}-2H-4v\\
&+V_{mn}\mu^{ijm}\mu^{kln}\nabla^2_{ik}v_{jl}\\
&+V_{11}\left(R_{2313}v_{12}-R_{2312}v_{13}\right)\\
&+V_{22}\left(R_{1323}v_{12}-R_{1321}v_{23}\right)\\
&+V_{33}\left(R_{1232}v_{31}-R_{1231}v_{23}\right).
\end{align}
\end{lemma}
\begin{proof}
The evolution of the Riemann curvature tensor is given by the standard formula

\begin{align*}
\partial_t R_{ijkl}=&\nabla_{il}^2h_{jk}+\nabla^2_{jk}h_{il}-\nabla^2_{ik}h_{jl}-\nabla^2_{jl}h_{ik}\\
&+\nabla_{il}^2v_{jk}+\nabla^2_{jk}v_{il}-\nabla^2_{ik}v_{jl}-\nabla^2_{jl}v_{ik}\\
&+g^{pq}(R_{ijpk}h_{ql}+R_{ijpl}h_{qk})+g^{pq}(R_{ijpk}v_{ql}+R_{ijpl}v_{qk}).
\end{align*}
On the other hand, the evolution of the volume form is given by $$\partial_t\mu_{ijk}=(H+v)\mu_{ijk},\quad\partial_t\mu^{ijk}=-(H+v)\mu^{ijk}.$$
Since $P^{mn}=-\frac{1}{4}\mu^{ijm}\mu^{kln}R_{ijkl}$, we obtain
\begin{align*}
\partial_tP^{mn}=&\mu^{ijm}\mu^{kln}\nabla^2_{ik}h_{jl}-\frac{1}{2}\mu^{ijm}\mu^{kln}g^{pq}R_{ijpl}h_{qk}-2HP^{mn}\\
&+\mu^{ijm}\mu^{kln}\nabla^2_{ik}v_{jl}-\frac{1}{2}\mu^{ijm}\mu^{kln}g^{pq}R_{ijpl}v_{qk}-2vP^{mn}.
\end{align*}
Now using the identity
\[\frac{1}{2}\mu^{ijm}\mu^{kln}g^{pq}R_{ijpl}h_{qk}+HP^{mn}=g^{mn}\det P,\]
we arrive at
\begin{align*}
\partial_tP^{mn}=&\nabla^2_{kl}(P^{kl}P^{mn}-P^{km}P^{ln})-g^{mn}\det P-HP^{mn}\\
&+\mu^{ijm}\mu^{kln}\nabla^2_{ik}v_{jl}-\frac{1}{2}\mu^{ijm}\mu^{kln}g^{pq}R_{ijpl}v_{qk}-2vP^{mn}.
\end{align*}

\begin{align*}
\partial_t u=& V_{ij}\left(\Box P^{ij}-\nabla_lP^{ik}\nabla_kP^{jl}-g^{ij}\det P -HP^{ij}\right)- g_{ij}\partial_t g^{ij}\\
&+V_{mn}(\mu^{ijm}\mu^{kln}\nabla^2_{ik}v_{jl}-\frac{1}{2}\mu^{ijm}\mu^{kln}g^{pq}R_{ijpl}v_{qk}-2vP^{mn})\\
=&V_{ij}P^{kl}\nabla_k\nabla_lP^{ij}- V_{ij}\nabla_lP^{ik}\nabla_kP^{jl}-2H\\
&+V_{mn}(\mu^{ijm}\mu^{kln}\nabla^2_{ik}v_{jl}-\frac{1}{2}\mu^{ijm}\mu^{kln}g^{pq}R_{ijpl}v_{qk}-2vP^{mn})\\
=&P^{kl}\nabla_k\left(V_{ij}\nabla_lP^{ij}\right)-P^{kl}\nabla_k V_{ij}\nabla_l P^{ij}-V_{ij}\nabla_lP^{ik}\nabla_kP^{jl}-2H\\
&+V_{mn}(\mu^{ijm}\mu^{kln}\nabla^2_{ik}v_{jl}-\frac{1}{2}\mu^{ijm}\mu^{kln}g^{pq}R_{ijpl}v_{qk}-2vP^{mn})\\
=&\Box u-P^{kl}\nabla_k V_{ij}\nabla_l P^{ij}+P^{jl}\nabla_kV_{ij}\nabla_lP^{ik}-2H\\
&+V_{mn}(\mu^{ijm}\mu^{kln}\nabla^2_{ik}v_{jl}-\frac{1}{2}\mu^{ijm}\mu^{kln}g^{pq}R_{ijpl}v_{qk})-6v.
\end{align*}
Furthermore,

\begin{align*}
\frac{1}{2}\mu^{ij1}\mu^{kl1}R_{ijpl}v_{pk}&=\mu^{kl1}R_{23pl}v_{pk}\\
&=\mu^{321}R_{23p2}v_{p3}+\mu^{231}R_{23p3}v_{p2}\\
&=R_{2323}v_{22}+R_{2313}v_{12}-R_{2312}v_{13}-R_{2332}v_{33}\\
&=-P^{11}(v_{22}+v_{33})+R_{2313}v_{12}-R_{2312}v_{13}.
\end{align*}
Thus
\[V_{11}\left(\frac{1}{2}\mu^{ij1}\mu^{kl1}R_{ijpl}v_{pk}\right)=-(v_{22}+v_{33})+V_{11}\left(R_{2313}v_{12}-R_{2312}v_{13}\right)\]
\[V_{22}\left(\frac{1}{2}\mu^{ij2}\mu^{kl2}R_{ijpl}v_{pk}\right)=-(v_{11}+v_{33})+V_{22}\left(R_{1323}v_{12}-R_{1321}v_{23}\right)\]
\[V_{33}\left(\frac{1}{2}\mu^{ij3}\mu^{kl3}R_{ijpl}v_{pk}\right)=-(v_{11}+v_{22})+V_{33}\left(R_{1232}v_{31}-R_{1231}v_{23}\right).\]

Consequently, we obtain
\begin{align*}
\partial_t u=&\Box u-P^{kl}\nabla_k V_{ij}\nabla_l P^{ij}+P^{jl}\nabla_kV_{ij}\nabla_lP^{ik}-2H-4v\\
&+V_{mn}\mu^{ijm}\mu^{kln}\nabla^2_{ik}v_{jl}\\
&+V_{11}\left(R_{2313}v_{12}-R_{2312}v_{13}\right)\\
&+V_{22}\left(R_{1323}v_{12}-R_{1321}v_{23}\right)\\
&+V_{33}\left(R_{1232}v_{31}-R_{1231}v_{23}\right).
\end{align*}
\end{proof}


A simple calculation shows that the time derivative of the Christoffel symbols is give by
\begin{align}
\partial_t \Gamma_{ij}^k&=g^{kl}\left(\nabla_ih_{jl}+\nabla_jh_{il}-\nabla_{l}h_{ij}\right)+g^{kl}\left(\nabla_iv_{jl}+\nabla_jv_{il}-\nabla_{l}v_{ij}\right).
\end{align}
We recall the following identities from Lemma 1 in Chow-Hamilton
\begin{align}
P^{ij}\nabla_ih_{jk}=\frac{1}{2}P^{ij}\nabla_kh_{ij}.
\end{align}
Using these identities we calculate
\begin{align}
P^{ij}\partial_t \Gamma_{ij}^k=&g^{kl}\left(P^{ij}\nabla_ih_{jl}+P^{ij}\nabla_jh_{il}-P^{ij}\nabla_{l}h_{ij}\right)\\
&+g^{kl}P^{ij}\left(\nabla_iv_{jl}+\nabla_jv_{il}-\nabla_{l}v_{ij}\right)\\
=&g^{kl}\left(\frac{1}{2}P^{ij}\nabla_lh_{ij}+\frac{1}{2}P^{ij}\nabla_lh_{ij}-P^{ij}\nabla_{l}h_{ij}\right)\\
&+g^{kl}P^{ij}\left(\nabla_iv_{jl}+\nabla_jv_{il}-\nabla_{l}v_{ij}\right)\\
=&g^{kl}P^{ij}\left(2\nabla_iv_{jl}-\nabla_{l}v_{ij}\right).
\end{align}
\begin{lemma}
 We have
\begin{align}
(\partial_t \Box-\Box\partial_t) u=&\left(\Box P^{mn}-\nabla_l P^{km}\nabla_lP^{kn}-g^{mn}\det P\right)\nabla^2_{mn}u\\
&+\left(\mu^{ijm}\mu^{kln}\nabla^2_{ik}v_{jl}-\frac{1}{2}\mu^{ijm}\mu^{kln}g^{pq}R_{ijpl}v_{qk}\right)\nabla^2_{mn}u\\
&-g^{kl}P^{ij}\left(2\nabla_iv_{jl}-\nabla_{l}v_{ij}\right)\nabla_ku-(H+2v)\Box u.
\end{align}
\end{lemma}
\begin{proof}
\begin{align*}
(\partial_t \Box-\Box \partial_t)u
=&\partial_tP^{ij}\cdot\nabla^2_{ij}u-P^{ij}\partial_t\Gamma_{ij}^k\cdot\nabla_ku\\
=&\left(\Box P^{mn}-\nabla_l P^{km}\nabla_lP^{kn}-g^{mn}\det P-HP^{mn}\right)\nabla^2_{mn}u\\
&+\left(\mu^{ijm}\mu^{kln}\nabla^2_{ik}v_{jl}-\frac{1}{2}\mu^{ijm}\mu^{kln}g^{pq}R_{ijpl}v_{qk}-2vP^{mn}\right)\nabla^2_{mn}u\\
&-g^{kl}P^{ij}\left(2\nabla_iv_{jl}-\nabla_{l}v_{ij}\right)\nabla_ku.
\end{align*}
\end{proof}


\end{document}
