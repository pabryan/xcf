\documentclass{amsart}
\usepackage[ocgcolorlinks,linktoc=all]{hyperref}
\usepackage{cancel}
\hypersetup{citecolor=blue,linkcolor=red}
\newtheorem{theorem}{Theorem}
\newtheorem*{thmA}{Theorem}
\newtheorem*{thmB}{Theorem}
\newtheorem*{rem}{Remark}
\newtheorem*{thmmain}{Theorem}
\newtheorem{lemma}[theorem]{Lemma}
\newtheorem{proposition}[theorem]{Proposition}
\newtheorem*{propmain}{Proposition}
\newtheorem{corollary}[theorem]{Corollary}
\theoremstyle{definition}
\newtheorem{definition}[theorem]{Definition}
\newtheorem{example}[theorem]{Example}
\newtheorem{xca}[theorem]{Exercise}

\theoremstyle{remark}
\newtheorem{remark}[theorem]{Remark}

\newcommand{\abs}[1]{\lvert#1\rvert}
\numberwithin{equation}{section}

\newcommand{\blankbox}[2]{%
  \parbox{\columnwidth}{\centering
    \setlength{\fboxsep}{0pt}%
    \fbox{\raisebox{0pt}[#2]{\hspace{#1}}}%
  }%
}

\begin{document}

\title[Cross curvature flow]
 {Cross curvature flow}

\curraddr{}
\email{}
\date{\today}

\dedicatory{}
\subjclass[2010]{}
\keywords{}

\begin{abstract}


\end{abstract}

\maketitle

\section{Set up}
The Riemannian metric is $g_{ij},$ its inverse is $g^{ij}$. The Levi-Civita connection is given by the Christoffel symbols


\begin{equation}
\Gamma_{ij}^k=\frac{1}{2}g^{kl}\left(\partial_ig_{jl}+\partial_jg_{il}-\partial_lg_{ij}\right),
\end{equation}
and the Riemannian curvature tensor is

\begin{equation}
R_{ijk}^l=\partial_i\Gamma_{jk}^l-\partial_j\Gamma_{ik}^l+\Gamma_{jk}^p\Gamma_{ip}^l-\Gamma_{ik}^p\Gamma_{jp}^l.
\end{equation}

We lower the index to the third position, so that
\begin{equation}
R_{ijkl}=g_{kp}R_{ijl}^p.
\end{equation}

The Ricci curvature is the contraction
\begin{equation}
R_{ij}=g^{kl}R_{ikjl}.
\end{equation}
The scalar curvature $R=g^{ij}R_{ij}.$

The evolution equation of the metric is given by

\begin{equation}
\partial_tg_{ij}=2h_{ij}
\end{equation}
where
\begin{equation}
h_{ij}:=-\frac{1}{2}P^{kl}R_{ikjl}
\end{equation}
and
\begin{equation}
P_{ij}:=(R_{ij}-\frac{1}{2}Rg_{ij}),\quad
P^{mn}=(R_{ij}-\frac{1}{2}Rg_{ij})g^{im}g^{jn}.
\end{equation}
We also write
\begin{equation}
P=g_{ij}P^{ij},\quad V=g^{ij}V_{ij},\quad H=g^{ij}h_{ij},\quad  T^{kij}=P^{kl}\nabla_l P^{ij}.
\end{equation}
We mention that
\begin{equation}
h_{ij}=V_{ij}\det P
\end{equation}
where $V_{ij}$ is the inverse of $P^{ij}.$
For simplicity we set
\[u:=\log\det p.\]
In the sequel, $\mu_{ijk}$ is the volume form and is nonzero only when $i,j,k$ are distinct, in which case it is sign of the permutation $(ijk)$. Using the volume form, we may express $P^{mn}$ as
\begin{equation}
P^{mn}=-\frac{1}{4}\mu^{ijm}\mu^{kln}R_{ijkl}.
\end{equation}
\section{Harnack quadratic}
A solution to cross curvature flow is a cross curvature soliton if there exists a vector field $V$ and $\lambda\in \mathbb{R}$ such that at some time
\begin{equation}
2h_{ij}+\nabla_iV_j+\nabla_jV_i=2\lambda g_{ij}.
\end{equation}
To obtain a suitable Harnack quadratic for XCF we follow Hamilton's procedure in obtaining his Harnack quadratic for the Ricci flow.

Suppose $V_i=\nabla_if.$ Then the soliton equation reads
\begin{equation}
h_{ij}+\nabla_iV_j=\lambda g_{ij}.
\end{equation}
We take the covariant derivative of both sides with respect to $\nabla_l:$
\begin{align*}
\nabla_lh_{ij}+\nabla_l\nabla_iV_j=0.
\end{align*}
We have
\begin{align*}
\nabla_l\nabla_iV_j-\nabla_i\nabla_lV_j={R_{lij}}^mV_m.
\end{align*}
Thus putting the last two identities together gives
\begin{align*}
\nabla_lh_{ij}-\nabla_ih_{lj}={R_{ilj}}^mV_m.
\end{align*}
We take the covariant derivative of both sides with respect to $\nabla_k:$
\begin{align*}
\nabla_k\nabla_lh_{ij}-\nabla_k\nabla_ih_{lj}=\nabla_k{R_{ilj}}^mV_m+{R_{ilj}}^m(\lambda g_{km}-h_{km}).
\end{align*}
Taking the trace with respect to $P^{kl}$ yields:
\begin{align*}
\Box h_{ij}-P^{kl}\nabla_k\nabla_ih_{lj}&=P^{kl}\nabla_k{R_{ilj}}^mV_m+P^{kl}{R_{ilj}}^m(\lambda g_{km}-h_{km})\\
&=P^{kl}(-\nabla^mR_{ilkj}-\nabla_j{{R_{il}}^m}_k)V_m+P^{kl}{R_{ilj}}^m(\lambda g_{km}-h_{km})\\
&=-P^{kl}\nabla^mR_{ilkj}V_m-P^{kl}\nabla_j{{R_{il}}^m}_kV_m-2\lambda h_{ij}-\det P R_{ij}.
\end{align*}
We may write
\[P^{kl}\nabla^mR_{ilkj}=2\nabla^mh_{ij}-R_{ilkj}\nabla^mP^{kl}\]
and
\[P^{kl}\nabla_j{{R_{il}}^m}_k=-2\nabla_jh_i^m-{{R_{il}}^m}_k\nabla_jP^{kl}.\]
Using these two identities we arrive at
\begin{align*}
\Box h_{ij}-P^{kl}\nabla_k\nabla_ih_{lj}=&-2(\nabla^mh_{ij}-\nabla_jh_i^m)V_m+(R_{ilkj}\nabla^mP^{kl}+{{R_{il}}^m}_k\nabla_jP^{kl})V_m\\
&-2\lambda h_{ij}-R_{ij}\det P .
\end{align*}
Moreover, we calculate
\begin{align*}
P^{kl}\nabla_k\nabla_ih_{lj}&=P^{kl}(\nabla_i\nabla_k h_{lj}+{R_{kil}}^mh_{mj}+{R_{kij}}^mh_{lm})\\
&=P^{kl}\nabla_i\nabla_k h_{lj}-2(h^2)_{ij}-R_{ij}\det P\\
&=\frac{1}{2}\nabla_i(P^{kl}\nabla_j h_{kl})-\nabla_iP^{kl}\nabla_k h_{lj}-2(h^2)_{ij}-R_{ij}\det P\\
&=\frac{1}{2}P^{kl}\nabla_i\nabla_j h_{kl}+\frac{1}{2}\nabla_iP^{kl}\nabla_j h_{kl}-\nabla_iP^{kl}\nabla_k h_{lj}-2(h^2)_{ij}-R_{ij}\det P.
\end{align*}
Consequently,
\begin{align*}
\Box h_{ij}=&-2(\nabla^mh_{ij}-\nabla_jh_i^m)V_m+(R_{ilkj}\nabla^mP^{kl}+{{R_{il}}^m}_k\nabla_jP^{kl})V_m\\
&-2\lambda h_{ij}-2R_{ij}\det P-2(h^2)_{ij}\\
&+\frac{1}{2}P^{kl}\nabla_i\nabla_j h_{kl}+\frac{1}{2}\nabla_iP^{kl}\nabla_j h_{kl}-\nabla_iP^{kl}\nabla_k h_{lj}.
\end{align*}
Taking the trace with respect to $P^{ij}$  gives
\begin{align*}
\frac{1}{2}P^{ij}\Box h_{ij}=&(2h_{ij}\nabla^mP^{ij}+P^{ij}{{R_{il}}^m}_k\nabla_jP^{kl}-P^{ij}\nabla^mh_{ij})V_m\\
&-6\lambda \det P+4H\det P-2H\det P+P^{ij}\left(\frac{1}{2}\nabla_iP^{kl}\nabla_j h_{kl}-\nabla_iP^{kl}\nabla_k h_{lj}\right).
\end{align*}
Note that for any choice of $\gamma$ we have
\[\nabla^mh_{ij}-\nabla_jh_i^m={{{R_j}^{m}}_{i}}^nV_n\Rightarrow \frac{\gamma}{2}P^{ij}\nabla^mh_{ij}V_m=-2\gamma h^{mn}V_nV_m.\]
Therefore,
\begin{align*}
\frac{1}{2}P^{ij}\Box h_{ij}=&\left(2h_{ij}\nabla^mP^{ij}+P^{ij}{{R_{il}}^m}_k\nabla_jP^{kl}+(\frac{\gamma}{2}-1)P^{ij}\nabla^mh_{ij}\right)V_m+2\gamma h^{mn}V_nV_m\\
&-6\lambda \det P+2H\det P+ P^{ij}\left(\frac{1}{2}\nabla_iP^{kl}\nabla_j h_{kl}-\nabla_iP^{kl}\nabla_k h_{lj}\right).
\end{align*}

\section{Basic evolution equations}
The following is the main result of the paper.
\begin{theorem}
\[\partial_t u-\frac{1}{2}P^{ij}\nabla_iu\nabla_ju+\frac{3}{2t}\geq 0. \]
\end{theorem}

\begin{lemma}
We have the following evolution equations:
\begin{align}
\partial_tP^{mn}=&\Box P^{mn}-\nabla_lP^{km}\nabla_kP^{ln}-g^{mn}\det P-HP^{mn}.
\end{align}
\begin{align}
\partial_t u=&\Box u+\left(P^{jl}\nabla_lP^{ik}-P^{kl}\nabla_l P^{ij}\right)\nabla_kV_{ij}-2H.
\end{align}
\end{lemma}
\begin{proof}
The evolution of the Riemann curvature tensor is given by the standard formula

\begin{align*}
\partial_t R_{ijkl}=&\nabla_{il}^2h_{jk}+\nabla^2_{jk}h_{il}-\nabla^2_{ik}h_{jl}-\nabla^2_{jl}h_{ik}\\
&+g^{pq}(R_{ijpk}h_{ql}+R_{ijpl}h_{qk}).
\end{align*}
On the other hand, the evolution of the volume form is given by $$\partial_t\mu_{ijk}=H\mu_{ijk},\quad\partial_t\mu^{ijk}=-H\mu^{ijk}.$$
Since $P^{mn}=-\frac{1}{4}\mu^{ijm}\mu^{kln}R_{ijkl}$, we obtain
\begin{align*}
\partial_tP^{mn}=&\mu^{ijm}\mu^{kln}\nabla^2_{ik}h_{jl}-\frac{1}{2}\mu^{ijm}\mu^{kln}g^{pq}R_{ijpl}h_{qk}-2HP^{mn}.
\end{align*}
Now using the identity
\[\frac{1}{2}\mu^{ijm}\mu^{kln}g^{pq}R_{ijpl}h_{qk}+HP^{mn}=g^{mn}\det P,\]
we arrive at
\begin{align*}
\partial_tP^{mn}=&\nabla^2_{kl}(P^{kl}P^{mn}-P^{km}P^{ln})-g^{mn}\det P-HP^{mn}.
\end{align*}

\begin{align*}
\partial_t u=& V_{ij}\left(\Box P^{ij}-\nabla_lP^{ik}\nabla_kP^{jl}-g^{ij}\det P -HP^{ij}\right)- g_{ij}\partial_t g^{ij}\\
=&V_{ij}P^{kl}\nabla_k\nabla_lP^{ij}- V_{ij}\nabla_lP^{ik}\nabla_kP^{jl}-2H\\
=&P^{kl}\nabla_k\left(V_{ij}\nabla_lP^{ij}\right)-P^{kl}\nabla_k V_{ij}\nabla_l P^{ij}-V_{ij}\nabla_lP^{ik}\nabla_kP^{jl}-2H\\
=&\Box u-P^{kl}\nabla_k V_{ij}\nabla_l P^{ij}+P^{jl}\nabla_kV_{ij}\nabla_lP^{ik}-2H.
\end{align*}
\end{proof}


A simple calculation shows that the time derivative of the Christoffel symbols is give by
\begin{align}
\partial_t \Gamma_{ij}^k&=g^{kl}\left(\nabla_ih_{jl}+\nabla_jh_{il}-\nabla_{l}h_{ij}\right).
\end{align}
We recall the following identities from Lemma 1 in Chow-Hamilton
\begin{align}
P^{ij}\nabla_ih_{jk}=\frac{1}{2}P^{ij}\nabla_kh_{ij}.
\end{align}
Using these identities we calculate
\begin{align*}
P^{ij}\partial_t \Gamma_{ij}^k=&g^{kl}\left(P^{ij}\nabla_ih_{jl}+P^{ij}\nabla_jh_{il}-P^{ij}\nabla_{l}h_{ij}\right)\\
=&g^{kl}\left(\frac{1}{2}P^{ij}\nabla_lh_{ij}+\frac{1}{2}P^{ij}\nabla_lh_{ij}-P^{ij}\nabla_{l}h_{ij}\right)=0.
\end{align*}
\begin{lemma}
 We have
\begin{align}
(\partial_t \Box-\Box\partial_t) u=&\partial_tP^{ij}\cdot\nabla^2_{ij}u.
\end{align}
\end{lemma}
\begin{proof}
\begin{align*}
(\partial_t \Box-\Box \partial_t)u
=&\partial_tP^{ij}\cdot\nabla^2_{ij}u-P^{ij}\partial_t\Gamma_{ij}^k\cdot\nabla_ku.
\end{align*}
\end{proof}
\begin{lemma}
\begin{align}
\partial_t\nabla_iu&=\Box\nabla_i u+\nabla_iP^{kl}\cdot \nabla_{kl}^2u+2h_i^m\nabla_mu\\
&+\nabla_i\left(\left(P^{ml}\nabla_lP^{mk}-P^{kl}\nabla_l P^{mn}\right)\nabla_kV_{mn}\right)-2\nabla_iH.\nonumber
\end{align}
\end{lemma}
\begin{proof}
\begin{align*}
\partial_t\nabla_iu&=\nabla_i(\Box u+\left(P^{ml}\nabla_lP^{mk}-P^{kl}\nabla_l P^{mn}\right)\nabla_kV_{mn}-2H)\\
&=\Box\nabla_i u+\nabla_iP^{kl}\cdot \nabla_{kl}^2u+P^{kl}{R_{ikl}}^m\nabla_mu\\
&+\nabla_i\left(\left(P^{ml}\nabla_lP^{mk}-P^{kl}\nabla_l P^{mn}\right)\nabla_kV_{mn}\right)-2\nabla_iH.
\end{align*}
Furthermore,
\[P^{kl}{R_{ikl}}^m=2h_{ij}g^{jm}.\]
\end{proof}
\begin{lemma}
\begin{equation}
\partial_tH=-2\|h\|^2-\frac{(h^2)_{mn}}{\det P}\partial_tP^{mn}+H\partial_tu.
\end{equation}
\end{lemma}
\begin{proof}
\begin{align*}
\partial_tH=&\partial_t(V\det P)\\
=&\det P\partial_t V+H\partial_tu\\
=&\det P(-2h^{ij}V_{ij}-g^{ij}V_{im}V_{jn}\partial_tP^{mn})+H\partial_tu\\
=&-2\|h\|^2-\frac{(h^2)_{mn}}{\det P}\partial_tP^{mn}+H\partial_tu.
\end{align*}
\end{proof}
\end{document}
